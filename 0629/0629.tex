\section{線形計画法}
\subsection{双対問題}

以下の問題を考える.
\begin{align}
  &{\mathrm{maximize}}\transpose{C}x\\
  &{\mathrm{subject to}}Ax\leq b,\ x\geq 0
\end{align}

今制約条件に$\transpose{y}>0$をかけることで$\transpose{y}Ax\leq\transpose{y}b$なのでもし$\transpose{y}Ax\geq\transpose{C}x$ならば$\transpose{C}x\leq\transpose{y}b$なので$\transpose{C}x\leq\transpose{y}b$である.
\begin{equation*}
  \begin{array}{cc}
    主問題&\\
    {\mathrm{max}}&\transpose{C}x\\
    {\mathrm{s.t.}}&Ax\leq b\\
    &x\geq 0
  \end{array}\leq
  \begin{array}{cc}
    双対問題&\\
    {\mathrm{min}}&\transpose{y}b\\
    {\mathrm{s.t.}}&\transpose{y}A\geq\transpose{c}\\
    &y\geq 0
  \end{array}
\end{equation*}
実は符号が成り立つと強双対性で不等号の場合は弱双対性
\subsection{Farkasの補題}
\begin{thm}
  $A,b$がgivenのとき\\
  \fbox{${}^{\exists}x\geq 0,\ Ax=b$}$\Leftrightarrow$\fbox{${}^{\forall}y\geq 0,\ \transpose{y}A\geq 0\Rightarrow\transpose{y}b\geq 0$}$\Leftrightarrow
  \begin{array}{c}
    \transpose{y}b<0\\
    \transpose{y}A\geq 0
  \end{array}$が解を持たない.
\end{thm}

$\Rightarrow$は容易である.左の項を満たす$\hat{x}$を用いて$\transpose{y}b=\transpose{y}(A\hat{x})=(\transpose{y}A)\hat{x}\geq 0({}^{\forall}y:\transpose{y}A\geq 0)$

$\Leftarrow$について.
\begin{align}
  (左の項)&\Leftrightarrow\displaystyle\sum_{j=1}^n x_j a_j =b\ \ \ x_j \geq 0\\
  &\Leftrightarrow b\in{\mathrm{cone}}(a_1 ,a_2 ,\cdots a_n )
\end{align}
ここでcone($a_1 ,a_2 $)とは$\bvector{a_1},\bvector{a_2}$のあいだに$\bvector{b}$が存在することを表す.

一方$\transpose{y}a_j \geq 0(j=1,2,\cdots n)\Rightarrow\transpose{y}b\geq 0$
\begin{thm}{Farkasの補題}
  \begin{align}
    \fbox{${}^{\exists}x\geq 0\ \ Ax\leq b$}&\Leftrightarrow y\geq 0,\ \transpose{y}A\geq 0\\
    &\Leftrightarrow
    \begin{cases}
      {}^{\exists}x\geq 0\\
      {}^{\exists}s\geq 0
    \end{cases}\ \ \ Ax+s=b\ (sはスラック変数)\\
    &\Leftrightarrow {}^{\exists}
    \begin{pmatrix}
      x\\
      s
    \end{pmatrix}\geq 0\ \ \
    \begin{pmatrix}
      A&I
    \end{pmatrix}
    \begin{pmatrix}
      x\\
      s
    \end{pmatrix}=b\ \ \ 等号なので上の定理のパターン\\
    &\Leftrightarrow\transpose{y}
    \begin{pmatrix}
      A&I
    \end{pmatrix}\geq 0\Leftrightarrow\transpose{y}b\geq 0
  \end{align}

\end{thm}
\subsection{強双対性}
主問題と双対問題に実行可能解が存在するならば,両者に最適解$x^{\ast},y^{\ast}$が存在し$\transpose{C}x^{\ast}=\transpose{y^{\ast}}b$である.

証明は弱双対性が成り立つので$\transpose{C}x^{\ast}\leq\transpose{y^{\ast}}b$が成り立つから以下を示せばいい.
\begin{equation*}
  \fbox{
  $\begin{array}{c}
    {}^{\exists}x (x\geq 0,\ Ax\leq b)\\
    {}^{\exists}y (y\geq 0,\ \transpose{y}A\geq c)
  \end{array}\ \ \ \transpose{C}x\geq\transpose{y}b$
  }
\end{equation*}
\begin{align}
  \fbox{\ }&\Leftrightarrow{}^{\exists}\
  \begin{pmatrix}
    x\\
    y
  \end{pmatrix}\geq 0,\
  \begin{pmatrix}
    A&0\\
    0&-\transpose{A}\\
    -\transpose{c}&\transpose{b}
  \end{pmatrix}
  \begin{pmatrix}
    x\\
    y
  \end{pmatrix}\leq
  \begin{pmatrix}
    b\\
    -c\\
    0
  \end{pmatrix}\\
  &\Leftrightarrow{\mathrm{Farkasの補題}}\fbox{
  $\begin{array}{c}
     a:\begin{pmatrix}
       u\\
       v\\
       \theta
     \end{pmatrix}\geq 0\
     \begin{pmatrix}
       \transpose{A}&0&-c\\
       0&-A&b
     \end{pmatrix}
     \begin{pmatrix}
       u\\
       v\\
       \theta
     \end{pmatrix}\geq 0\\
     \Downarrow\\
     b:\ \begin{pmatrix}
       \transpose{b}&-c&0
   \end{pmatrix}
   \begin{pmatrix}
     u\\
     v\\
     \theta
   \end{pmatrix}\geq 0
   \end{array}
  $}を示せばいい.
\end{align}

\begin{equation*}
  \begin{cases}
    a:\ u,\ v,\ \theta\geq 0,\ \transpose{A}u\geq c\theta ,\ Av\leq b\theta\\
    b:\ \transpose{b}u\geq\transpose{c}v
  \end{cases}
\end{equation*}
$\theta =0$と$\theta >0$で場合分けする.

$\theta >0$のときは
\begin{equation*}
  \transpose{b}u\geq \transpose{\left( \displaystyle\frac{1}{\theta}Av\right)}u=\frac{1}{\theta}\transpose{v}\transpose{A}u\geq\frac{1}{\theta}\transpose{v}(c\theta )=\transpose{v}c
\end{equation*}

$\theta =0$のとき示すべきは$\begin{pmatrix}u\\v\end{pmatrix}$である.
\begin{align}
  &\begin{pmatrix}
    \transpose{u}&\transpose{v}
  \end{pmatrix}
  \begin{pmatrix}
    A&0\\
    0&-\transpose{A}
  \end{pmatrix}\geq 0\\
  \Leftrightarrow &\begin{pmatrix}
    \transpose{u}&\transpose{v}
  \end{pmatrix}
  \begin{pmatrix}
    b\\
    -c
  \end{pmatrix}\geq 0\\
  \Leftrightarrow &{}^{\exists}\begin{pmatrix}
    x\\
    y
\end{pmatrix}\geq 0\
\begin{pmatrix}
  A&0\\
  0&-\transpose{A}
\end{pmatrix}
\begin{pmatrix}
  x\\
  y
\end{pmatrix}\leq
\begin{pmatrix}
  b\\
  -c
\end{pmatrix}
\end{align}
最後の変形はFarkasの補題を用いた.

