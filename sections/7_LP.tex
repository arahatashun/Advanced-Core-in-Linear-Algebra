\section{線形計画法}
\subsection{双対問題}

以下の問題を考える.
\begin{align}
  &{\text{maximize }}\transpose{c}x\\
  &{\text{subject to }}Ax\leq b,\ x\geq 0
\end{align}

今制約条件に$\transpose{y}>0$をかけることで$\transpose{y}Ax\leq\transpose{y}b$なのでもし$\transpose{y}Ax\geq\transpose{c}x$ならば$\transpose{c}x\leq\transpose{y}b$なので$\transpose{c}x\leq\transpose{y}b$である.
\begin{equation}
  \begin{array}{cc}
    主問題&\\
    {\mathrm{max}}&\transpose{c}x\\
    {\mathrm{s.t.}}&Ax\leq b\\
    &x\geq 0
  \end{array}\leq
  \begin{array}{cc}
    双対問題&\\
    {\mathrm{min}}&\transpose{y}b\\
    {\mathrm{s.t.}}&\transpose{y}A\geq\transpose{c}\\
    &y\geq 0
  \end{array}
\end{equation}
実はゆるい仮定のもとで符号が成り立つ(強双対性).
不等号の場合は弱双対性と呼ぶ.
\subsection{Farkasの補題}
\begin{itembox}[l]{Farkasの補題}
次の2条件は同値
\begin{enumerate}
    \item \(Ax=b\)を満たす非負べクトル$x\geq 0$が存在する.
    \item 任意の実数ベクトル$y$について,$y^T A \geq 0^T$ならば,
    \(y^T b \geq 0\)である.
\end{enumerate}
\end{itembox}

最後の条件はつまり
 \(\transpose{y}b<0\)
かつ\(\transpose{y}A\geq 0\)
が解を持たない.\\
$\Rightarrow$は容易である.左の項を満たす$\hat{x}$を用いて$\transpose{y}b=\transpose{y}(A\hat{x})=(\transpose{y}A)\hat{x}\geq 0({}^{\forall}y:\transpose{y}A\geq 0)$

$\Leftarrow$について.
\begin{align}
  (左の項)&\Leftrightarrow\displaystyle\sum_{j=1}^n x_j a_j =b\ \ \ x_j \geq 0 \quad(a_j\text{は列
  ベクトル})\\
  &\Leftrightarrow b\in{\mathrm{cone}}(a_1 ,a_2 ,\cdots a_n )
\end{align}
ここで\( b \in \) cone($a_1 ,a_2 $)とは$\bvector{a_1},\bvector{a_2}$のあいだに$\bvector{b}$が存在することを表す.
条件(1)が不成立のとき,つまり
$b \not \in \mathrm{cone}(a_1,\cdots,a_n)$
のときは,ある超平面でCone($A$)と$b$を分離することができる.
したがって$\transpose{y}a_j \geq 0(j=1,2,\cdots n)\Rightarrow\transpose{y}b\geq 0$.
したがって$b$は$a_j$のなす錐に含まれる.
\begin{itembox}[l]{Farkasの補題(2)}
等号が不等号に変わっていることに注意.次の2条件は同値.
\begin{enumerate}
    \item $Ax \leq b$を満たすベクトル$x \geq 0$が存在する.
    \item 任意の実数ベクトル$y$について,$y \geq 0$かつ$y^T A \geq  0^T $
    ならば$y^T b \geq 0 $
\end{enumerate}
\end{itembox}
 \begin{align}
    \fbox{${}^{\exists}x\geq 0\ \ Ax\leq b$}
    &
    \Leftrightarrow
    \fbox{ $ \forall y\geq 0,\ \transpose{y}A\geq 0
    \Rightarrow y^Tb\geq 0$}
\end{align}
\begin{align}
\because 
    &\Leftrightarrow
    \begin{cases}
      {}^{\exists}x\geq 0\\
      {}^{\exists}s\geq 0
    \end{cases}\ \ \ Ax+s=b\ (sはスラック変数)\\
    &\Leftrightarrow {}^{\exists}
    \begin{bmatrix}
      x\\
      s
    \end{bmatrix}\geq 0\ \ \
    \begin{bmatrix}
      A&I
    \end{bmatrix}
    \begin{bmatrix}
      x\\
      s
    \end{bmatrix}=b\ \ \ 等号なので上の補題のパターン\\
    &\Leftrightarrow
    \forall y \geq 0,\quad 
    \transpose{y}
    \begin{bmatrix}
      A&I
    \end{bmatrix}\geq 0\Rightarrow\transpose{y}b\geq 0
  \end{align}

\subsection{強双対性}
\begin{itembox}[l]{強双対定理}
(Primal)
\begin{align}
    \text{maximize } c^Tx\\
    \text{subject to } Ax\leq b \quad x \geq 0
\end{align}
(Dual)
\begin{align}
    \text{minimize } y^T b\\
    \text{subject to }y^T A \geq c^T \quad y\geq 0
\end{align}
主問題と双対問題に実行可能解が存在するならば,両者に最適解$x^{\ast},y^{\ast}$が存在し
\begin{equation}
    \transpose{c}x^{\ast}=\transpose{y^{\ast}}b
\end{equation}
である.
\end{itembox}
\begin{proof}
弱双対性が成り立つ
つまり$\transpose{c}x^{\ast}
\leq\transpose{y^{\ast}}b$が成り立つので強
双対性を示す.
\begin{itembox}[l]{強双対性}

$ {}^{\exists}x (x\geq 0,\ Ax\leq b) , {}^{\exists}y (y\geq 0,\ \transpose{y}A\geq c)$ satisfying
\begin{equation}
  \transpose{c}x\geq\transpose{y}b
\end{equation}
\end{itembox}
\begin{align}
&\Leftrightarrow{}^{\exists}\
  \begin{bmatrix}
    x\\
    y
  \end{bmatrix}\geq 0,\
  \begin{bmatrix}
    A&0\\
    0&-\transpose{A}\\
    -\transpose{c}&\transpose{b}
  \end{bmatrix}
  \begin{bmatrix}
    x\\
    y
  \end{bmatrix}\leq
  \begin{bmatrix}
    b\\
    -c\\
    0
  \end{bmatrix}\\
  &\Leftrightarrow{\mathrm{Farkasの補題}}\\
  &\fbox{
  $\begin{array}{c}
     a:\begin{bmatrix}
       u\\
       v\\
       \theta
     \end{bmatrix}\geq 0
     \quad 
     \begin{bmatrix}
       \transpose{A}&0&-c\\
       0&-A&b
     \end{bmatrix}
     \begin{bmatrix}
       u\\
       v\\
       \theta
     \end{bmatrix}\geq 0\\
     \Downarrow\\
     b:\ \begin{bmatrix}
       \transpose{b}&-\transpose{c} &0
   \end{bmatrix}
   \begin{bmatrix}
     u\\
     v\\
     \theta
   \end{bmatrix}\geq 0
   \end{array}
  $}を示せばいい.
\end{align}

\begin{equation*}
  \begin{cases}
    a:\ u,\ v,\ \theta\geq 0,\ \transpose{A}u\geq c\theta ,\ Av\leq b\theta\\
    b:\ \transpose{b}u\geq\transpose{c}v
  \end{cases}
\end{equation*}
$\theta =0$と$\theta >0$で場合分けする.

$\theta >0$のとき,
(a) から(b)を導く
\begin{equation}
  \transpose{b}u\geq \transpose{\left( \displaystyle\frac{1}{\theta}Av\right)}u=\frac{1}{\theta}\transpose{v}\transpose{A}u\geq\frac{1}{\theta}\transpose{v}(c\theta )=\transpose{v}c
\end{equation}

$\theta =0$のとき示すべきは
以下の通り.\\
\begin{align}
  &
  \begin{bmatrix}
  u\\v
  \end{bmatrix}\geq 0
  \quad 
  \begin{bmatrix}
    \transpose{u}&\transpose{v}
  \end{bmatrix}
  \begin{bmatrix}
    A&0\\
    0&-\transpose{A}
  \end{bmatrix}\geq 0\\
  \Rightarrow &\begin{bmatrix}
    \transpose{u}&\transpose{v}
  \end{bmatrix}
  \begin{bmatrix}
    b\\
    -c
  \end{bmatrix}\geq 0
  \end{align}
  Farkasの補題を用いて,
主問題の形に戻すと
  \begin{align}
  \Leftrightarrow &{}^{\exists}\begin{bmatrix}
    x\\
    y
\end{bmatrix}\geq 0\
\begin{bmatrix}
  A&0\\
  0&-\transpose{A}
\end{bmatrix}
\begin{bmatrix}
  x\\
  y
\end{bmatrix}\leq
\begin{bmatrix}
  b\\
  -c
\end{bmatrix}
\end{align}
最後の形から実行可能解を持つならば,
$C^Tx \geq y^Tb$がなりたつ$x,y$
が存在する.

\end{proof}