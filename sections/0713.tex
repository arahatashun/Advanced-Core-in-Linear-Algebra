\section{線形システム}
\subsection{初期値問題}
$\dot{x}(t)=ax(t),\ x(0)=x^0$の解は$x(t)=e^{at}x^0 (t\geq 0)$である.複数の変数に対して同様に以下の線形微分方程式系を考えると解は$x(t)=e^{tA}x^0 (t\geq 0)$となる.
\begin{equation}
  \dot{x}(t)=Ax(t),\ x(0)=x^0 (A\in\realnspace{n\times n})
\end{equation}
\begin{dfn}{行列の指数関数}
  $e^A = \displaystyle\sum_{k=0}^{\infty}\frac{1}{k!}A^k$
\end{dfn}
\begin{thm}
  $\lim_{r\to\infty}\displaystyle\sum_{k=0}^r \frac{1}{k!}A^k$は収束する
\end{thm}
なぜなら$S_r = \displaystyle\sum_{k=0}^r \frac{1}{k!}A^k$とすると
\begin{equation}
  \normsuffix{e^A -S_r}{}=\normsuffix{\displaystyle\sum_{k=r+1}^{\infty}\frac{1}{k!}A^k}{}\leq\sum_{r=k+1}^{\infty}\frac{1}{k!}A^k = e^{\normsuffix{A}{}}-\sum_{k=0}^r \frac{1}{k!}{\normsuffix{A}{}}^k \to 0\ (\normsuffix{PQ}{}\leq\normsuffix{P}{}\normsuffix{Q}{})
\end{equation}
\begin{thm}
  $e^{tA}=\displaystyle\int_0^t Ae^{\tau A}\diff\tau +I$\ (したがって$\displaystyle\frac{\diff}{\diff t}e^{At}=Ae^{tA}=e^{tA}A$)
\end{thm}

証明は以下.
\begin{align}
  \displaystyle\int_0^t Ae^{\tau A}\diff\tau &=\int_0^t A\left( \sum_{k=0}^{\infty}\frac{1}{k!}{(\tau A)}^k\right) \diff\tau\\
  &=\displaystyle\int_0^t \sum_{k=0}^{\infty}{\tau}^k \frac{1}{k!}A^{k+1}\diff\tau\\
  &=\displaystyle\sum_{k=0}^{\infty}\frac{t^{k+1}}{k+1}\frac{A^{k+1}}{k!}\\
  &=\displaystyle\sum_{k=0}^{\infty}\frac{1}{(k+1)!}{(tA)}^{k+1}\\
  &=\displaystyle\sum_{k'=0}^{\infty}\frac{1}{k'!}{(tA)}^{k'}-I\left( \frac{1}{0!}{(tA)}^0 \right)
\end{align}
\subsection{AのJordan標準形}
正則行列$S$を用いて$J=\inverse{S}AS$を考えると
\begin{align}
  &\inverse{S}e^{tA}S=\displaystyle\sum_{k=0}^{\infty}\frac{1}{k!}\inverse{S}{(tA)}^k S=\sum_{k=0}^{\infty}\frac{1}{k!}t^k {(\inverse{S}AS)}^k = e^{tJ}\\
  &e^{tA}=Se^{tJ}\inverse{S}
\end{align}

$A=\lambda\unitmatrix +J_4 (0)=
\begin{pmatrix}
  \lambda&1&0&0\\
  0&\lambda&1&0\\
  0&0&\lambda&1\\
  0&0&0&\lambda
\end{pmatrix}$を考えてみる.
\begin{align}
  e^{tA}&=e^{t\lambda I}e^{tJ_4 (0)}\\
  &=(e^{t\lambda}\unitmatrix )e^{tJ_4 (0)}\\
  e^{tJ_4 (0)}&=\displaystyle\sum_{k=0}^3 \frac{1}{k!}t^k {J_4 (0)}^k\\
  &=\unitmatrix +tJ_4 (0)+\displaystyle\frac{1}{2}t^2 {J_4 (0)}^2 +\frac{1}{6}t^3 {J_4 (0)}^3
\end{align}
\subsection{安定性}
${}^{\forall}x^0 \in\realnspace{1}$に対して$e^{tA}x^0 \to 0\Leftrightarrow{\mathrm{Re}}\lambda <0$が$A$のすべての固有値について成り立つとき$A$は安定であるという.
\begin{thm}
  $A(n\times n実行列)$が安定(すべての固有値について${\mathrm{Re}}\lambda <0$)\\
  $\Leftrightarrow Y(n\times n)$について
  \begin{align}
    &Y=\transpose{Y}>0\ (正定値)\\
    &YA+\transpose{A}Y<0\ (負定値)
  \end{align}
\end{thm}
負定値の場合の不等式をLyaponov不等式という.

$\Leftarrow$は容易で$\lambda$を$A$の固有値とすると
\begin{align}
  &Av=\lambda v(v\neq 0)\\
  &v^{\ast}\transpose{A}=\overline{\lambda}v^{\ast}\ (共役転置)
\end{align}

$YA+\transpose{A}Y<0$なので
\begin{align}
  &v^{\ast}(YA+\transpose{A}Y)v<0\\
  &v^{\ast}Y\underbrace{Av}_{\lambda v}+\underbrace{v^{\ast}\transpose{A}}_{\overline{\lambda}v^{\ast}}Yv<0\\
  &(\lambda +\overline{\lambda})v^{\ast}Yv<0\\
  よって&2{\mathrm{Re}}\lambda = \lambda +\overline{\lambda}<0
\end{align}

$\Rightarrow$について${\mathrm{Re}}\lambda <0$なので$e^{tA}\to 0(t\to\infty )$である.

正定値実対称行列$Q>0$を用いて以下のように置けばよい.
\begin{equation}
  Y=\displaystyle\int_0^{\infty}e^{t\transpose{A}}Qe^{tA}\diff t>0
\end{equation}
というのも
\begin{align}
  &\displaystyle\frac{\diff}{\diff t}(e^{t\transpose{A}}Qe^{tA})=e^{t\transpose{A}}Qe^{tA}A+\transpose{A}e^{e\transpose{A}}Qe^{tA}\\
  積分して&\displaystyle\int_0^{\infty}\frac{\diff}{\diff t}(e^{t\transpose{A}}Qe^{tA})=YA+\transpose{A}Y\\
  左辺は&\left[e^{t\transpose{A}}Qe^{tA}\right]_0^{\infty}=0-Q<0
\end{align}

任意の$Q>0$を与えてLyaponov方程式$YA+\transpose{A}Y+Q=0$を解いて($Y:$実対称行列が未知数)解$Y$が正定値なら$A$は安定である.
\subsection{Lyaponov方程式の可解性}
以下の方程式をSylvester方程式と呼ぶ.
\begin{equation}
  AX-XB=C\ \ (行列X\ m\times nが未知数)
\end{equation}
ただし$A:m\times m,\ B:n\times n,\ C:m\times n$

$X=[x_1 |x_2 |\cdots x_n]\to \tilde{x}=
\begin{bmatrix}
  x_1\\
  x_2\\
  \vdots\\
  x_n
\end{bmatrix}(mn\times 1行列)$とおく.Sylvester方程式は以下のように書ける.
\begin{align}
  &AX-XB=C\\
  &\left(
  \begin{bmatrix}
    A&&&0\\
    &A&&\\
    &&\ddots&\\
    0&&&A
  \end{bmatrix}-
  \begin{bmatrix}
    b_{11}\unitmatrix&b_{21}\unitmatrix&&0\\
    b_{12}\unitmatrix&b_{22}\unitmatrix&&\\
    &&\ddots&\\
    0&&&b_{nn}\unitmatrix
  \end{bmatrix}\right) \tilde{x}=
  \begin{bmatrix}
    c_1\\
    c_2\\
    \vdots\\
    c_n
  \end{bmatrix}=\tilde{c}
\end{align}

つまり$(T_n \otimes A-\transpose{B}\otimes I_m )\tilde{x}=\tilde{c}$である.$\otimes$はKronecker積を表し
\begin{align}
  &P:m\times n行列,\ \ \ Q:k\times l行列\\
  &P\otimes Q=
  \begin{pmatrix}
    P_{11}Q&P_{12}Q&&P_{1n}Q\\
    P_{21}Q&P_{22}Q&&\\
    &&\ddots&\\
    P_{m1}Q&&&P_{mn}Q
  \end{pmatrix}
\end{align}
\begin{align}
  &AX-XB=Cが任意のCに対して解を持つ\\
  \Leftrightarrow&(I_n \otimes A-\transpose{B}\otimes I_m )が正則\\
  \Leftrightarrow&(I_n \otimes A-\transpose{B}\otimes I_m )が0を固有値に持たない\\
  \Leftrightarrow&(実は)Aと\transpose{B}が同じ固有値を持たない
\end{align}

(実は)の証明は以下.

\begin{equation}
  \begin{cases}
    Au=\alpha u\\
    \transpose{B}v=\beta v
  \end{cases}
  \ \ \ とするとv\otimes u=
  \begin{bmatrix}
    v_1 u\\
    v_2 u\\
    \vdots\\
    v_n u
  \end{bmatrix}
\end{equation}
\begin{align}
  &(I_n \otimes A)(v\otimes u)=v\otimes (Au)=\alpha (v\otimes u)\\
  &(\transpose{B}\otimes I_m )(v\otimes u)=(\transpose{B}v)\otimes u=\beta (v\otimes u)\\
  したがって&(I_n \otimes A-\transpose{B}\otimes I_m)=\left\{ \alpha -\beta\Biggr|
  \begin{array}{c}
    \alpha はAの固有値\\
    \beta は\transpose{B}の固有値
  \end{array}\right\}
\end{align}

Lyaponov方程式に戻ると$YA+\transpose{A}Y+Q=0$でSylvester方程式との対応は$A\leftarrow\transpose{A},\ B\leftarrow -A$なので$\transpose{A}と-A$が同じ固有値を持たない$\Rightarrow{}^{\forall}Q:$Lyaponov方程式は可解

Sylvester方程式$AX-XB=C$について$A$が安定なら${\mathrm{Re}}\lambda <0\to \transpose{A}と-A$は同じ固有値を持たない.