\section{過去問}
\subsection{2018年期末}
\begin{itembox}[l]{問1}
$A,B$を$n$次の実対称正定値行列とする.
$A-B$が半正定値行列であるならば,$
\det A \geq \det B$が成り立つことを示せ.
\end{itembox}
\begin{equation}
    A - B \succeq 0 \text{なので} A \succeq B
\end{equation}

$A$,$B$は実対称ゆえ、
Courant Fisherの定理より
\begin{equation}
    \lambda_k(A) \geq \lambda_k(B)
\end{equation}
$B$は正定値行列であるので
\begin{equation}
    \lambda_k(A) \geq \lambda_k(B) > 0
\end{equation}
したがって
\begin{equation}
    \Pi_k \lambda_k(A) \geq  \Pi_k \lambda_k(B)
\end{equation}
行列式は固有値の積に等しいので
示された.
\begin{itembox}[l]{問2}
行列$A\in \mathbb{R}^{m \times n}, C \in \mathbb{R}^{k\times n}$
とベクトル$b \in \mathbb{R}^m, d \in \mathbb{R}^k$が与えられてる.
Farkasの補題または線形計画法の双対定理を用いて.以下の問に答えよ.
\begin{enumerate}
    \item 次の条件が同値であることを示せ.
    \begin{enumerate}
        \item $Ax \leq b$を満たす実数ベクトル$x$が存在する.
        \item 任意の非負ベクトル$y\geq 0$について,
        $A^T y =0$ならば$b^T y\geq 0$である.
    \end{enumerate}
    \item 次の条件が同値であることを示せ.
    \begin{enumerate}
        \item $Ax \leq b$かつ$Cx<d$を満たす実数ベクトル$x$が存在する.
        \item 任意の非負ベクトル$y\geq 0,z\geq 0$について,
        \begin{enumerate}
            \item $A^T y + C^Tz = 0$ならば,$b^T y + d^T z \geq 0$であり,かつ,
            \item $A^T y + C^T z = 0$かつ$z\neq 0$ならば$b^T y + d^T z >0$である.
        \end{enumerate}
    \end{enumerate}
\end{enumerate}
\end{itembox}
\begin{enumerate}

\item (b)$\Rightarrow$(a).
(b)の条件は次のように書き直すことができる.
\begin{equation}
    \forall y \quad \begin{bmatrix}
    A^T \\
    -A^T\\
    I_m
    \end{bmatrix}
    y \geq 0
    \Rightarrow
    b^T y \geq 0
\end{equation}
Farkas lemmaから
\begin{equation}
    \exists x^\ast \geq 0 \quad 
     \begin{bmatrix}
    A &
    -A &
    I_m
    \end{bmatrix}x^\ast  = b
\end{equation}
任意の$x^\ast \geq 0$に対して,
実数ベクトル$x$を
\begin{equation}
    x =x^\ast_{n1} -  x^\ast_{n2}
\end{equation}
とすることで.以下を満たすように構成できる.
\begin{equation}
    Ax \leq  \begin{bmatrix}
    A &
    -A &
    I_m
    \end{bmatrix}x^\ast 
    =\begin{bmatrix}
    A &
    -A &
    I_m
    \end{bmatrix}
    \begin{bmatrix}
    x^\ast_{n1}\\
    x^\ast_{n2}\\
    x^\ast_m
    \end{bmatrix}
    = b
\end{equation}
(a)$\Rightarrow$(b).非負ベクトル
$y^T$を(a)に左からかけて,
\begin{align}
    y^T Ax &\leq y^T b\\
    x^T A^T y &\leq b^T y
\end{align}
したがって$A^Ty=0$ならば$ b^T y \geq 0$である.


\item 
(a)$\Rightarrow$ (b).
(a)を転置して,$y\geq 0, z \geq 0$を右からかけると
\begin{equation}
    x^T A^T y \leq b^T y \quad x^T C^T z \leq d^T z
\end{equation}
足し合わせることで
\begin{equation}
    x^T (A^T y + C^Tz) \leq b^T y + d^T z
\end{equation}
したがって,(b)(i)が示された.さらに
先程,$z\geq0 $をかけたところを$z >0$に変えることで,
\begin{equation}
     x^T A^T y \leq b^T y \quad x^T C^T z < d^T z
\end{equation}
これを足し合わせることで
\begin{equation}
 x^T (A^T y + C^Tz) < b^T y + d^T z
 \end{equation}
 したがって(b)(ii)が示された.\\
 (b)$\Rightarrow$(a)
(b)(i)の条件を書き換えると
\begin{equation}
    \forall y \geq 0, \forall z \geq 0\quad
    \begin{bmatrix}
    y^T & z^T
    \end{bmatrix}
    \begin{bmatrix}
    A \\
    C
    \end{bmatrix}
    = 0 
    \Rightarrow
        \begin{bmatrix}
    y^T & z^T
    \end{bmatrix}
    \begin{bmatrix}
    b\\
    d
    \end{bmatrix}
    \geq 0
\end{equation}
(1)を用いるとこの条件から
\begin{equation}
     \begin{bmatrix}
    A \\
    C
    \end{bmatrix}
    x^\ast \leq    \begin{bmatrix}
    b\\
    d
    \end{bmatrix}
\end{equation}
したがって,
$Ax\leq b$かつ$Cx \leq d$を満たす実ベクトル$x
$が存在する.
\end{enumerate}
\begin{itembox}[l]{問題3}
$a$を整数とする.方程式
\begin{equation}
    \begin{bmatrix}
    15 & 3 & 6\\
    50 & 9 & 23
    \end{bmatrix}
    \begin{bmatrix}
    x_1\\
    x_2\\
    x_3
    \end{bmatrix}
    = 
    \begin{bmatrix}
    2 - a\\
    6 + 2a
    \end{bmatrix}
\end{equation}
が整数解($x_1,x_2,x_3$)を持つような
$a$を求めよ.また,
その時の解($x_1,x_2,x_3$)を求めよ.
\end{itembox}
Hermite標準形にする
\begin{align}
\begin{bmatrix}
15 & 3 & 6\\
50 & 9 & 23
\end{bmatrix}
\xrightarrow[]{\text{1列と2列を入れ替え}}
\begin{bmatrix}
15 & 6 & 3\\
50 & 23 & 9 
\end{bmatrix}
\xrightarrow[]{\text{2列から3列を2倍して引く}}
\begin{bmatrix}
15 & 0 & 3\\
50 & 5 & 9 
\end{bmatrix}\\
\xrightarrow[]{\text{2列と3列を入れ替え}}
\begin{bmatrix}
15 & 3 & 0\\
50 & 9 & 5 
\end{bmatrix}
\xrightarrow[]{\text{1列から2列を5倍して引く}}
\begin{bmatrix}
0 & 3 & 0\\
5 & 9 & 5 
\end{bmatrix}\\
\xrightarrow[]{\text{1列と2列を入れ替え}}
\begin{bmatrix}
 3 & 0 & 0\\
 9 & 5  & 5
\end{bmatrix}
\xrightarrow[]{\text{3列から2列を引く}}
\begin{bmatrix}
 3 & 0 & 0\\
 9 & 5  & 0
\end{bmatrix}\\
\xrightarrow[]{\text{1列から2列を引く}}
\begin{bmatrix}
 3 & 0 & 0\\
 4 & 5  & 0
\end{bmatrix}
\end{align}
単模行列$V$
を用いて
$Ax=b$を$AV(V^{-1})x=b$の形にする.
単模行列より$V^{-1}x$が整数解のときに
$x$も整数解となる.
\begin{align}
&
\begin{bmatrix}
15 & 3 & 6\\
50 & 9 & 23
\end{bmatrix}
\begin{bmatrix}
-1 & 1 & -1\\
6 & -5 & 3\\
0 & 0 & 1
\end{bmatrix}
=
\begin{bmatrix}
 3 & 0 & 0\\
 4 & 5  & 0
\end{bmatrix}
\end{align}
よって
\begin{align}
    \begin{bmatrix}
 3 & 0 & 0\\
 4 & 5  & 0
\end{bmatrix}
\begin{bmatrix}
 x^\ast_1\\
  x^\ast_2\\
   x^\ast_3
\end{bmatrix}
=   \begin{bmatrix}
    2 - a\\
    6 + 2a
\end{bmatrix}\\
 x^\ast_1 = \frac{2-a}{3}\\
 4 x^\ast_1 + 5  x^\ast_2 = 6+2a\\
  x^\ast_2 = \frac{2}{3}(a+1)
\end{align}
整数解であるためには
$2-a$が3でわりきれればよい.
$k$を整数として
$2-a = 3k$とすると$a = -3k+2$.
\begin{align}
    x^\ast_1 &= k\\
    x^\ast_2 &= -2k+2
\end{align}
$x^\ast_3 = l$という整数とすると
また$V$から
\begin{align}
\begin{bmatrix}
 x_1\\
 x_2\\
 x_3
\end{bmatrix}
= V \begin{bmatrix}
 x^\ast_1\\
 x^\ast_2\\
 x^\ast_3
\end{bmatrix}
=\begin{bmatrix}
 -1 & 1 & -1\\
6 & -5 & 3\\
0 & 0 & 1
\end{bmatrix}
\begin{bmatrix}
 k\\
-2k+2\\
l
\end{bmatrix}
\end{align}
\begin{align}
    x_1 &= -3k+2-l\\
    x_2 &= k -2l + 15k- 10 +5l
    = 16k +3l - 10 
\end{align}
まとめると,整数$k,l$を用いて
\begin{align}
    x_1 &=  -3k+2-l\\
    x_2 &= 16k +3l - 10 \\
    x_3 &= l
\end{align}