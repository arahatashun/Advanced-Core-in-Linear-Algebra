\subsection{特異値分解}
正方行列は場合によっては対角化できたが,正方行列でない行列でも対角化のようなことがしたい.そのようなときは特異値分解を行う.

$m\times n$行列$A$について直交行列(転置行列と逆行列が等しい)$U(m\times m),V(n\times n)$をとってくると
\begin{equation}
  \transpose{U}AV=
  \begin{array}{c|c}
    \begin{array}{ccc}
      {\sigma}_1 &&\\
      &\ddots&\\
      &&{\sigma}_r
    \end{array}&0\\\hline
    0&0
  \end{array}=\Sigma
\end{equation}
ここで$r=\mathrm{rank}A$であり$\Sigma$は特異値行列である.
\begin{align}
  &A=U\Sigma\transpose{V}\\
  &\transpose{A}=V\transpose{\Sigma}\transpose{U}\\
  &A\transpose{A}=U(\Sigma\transpose{\Sigma})\transpose{U}\\
  &\Sigma\transpose{\Sigma}(m\times m)=
  \begin{array}{|c|c|}
    \hline
    \begin{array}{ccc}
      {{\sigma}_1}^2 &&\\
      &\ddots&\\
      &&{{\sigma}_r}^2
    \end{array}&0\\\hline
    0&0\\\hline
  \end{array}
\end{align}
\begin{align}
  Uの列ベクトルは&A\transpose{A}の固有ベクトル\\
  {{\sigma}_j}^2は&A\transpose{A}の固有値\\
  Vの列ベクトルは&\transpose{A}Aの固有ベクトル\\
  {{\sigma}_j}^2は&\transpose{A}Aの固有値
\end{align}
\subsection{Jordan標準形}
\subsubsection{はじめに}
$A=\fourmatrix{0}{1}{0}{0}\ \ \ B=\fourmatrix{0}{0}{0}{0}$とする.
\begin{align}
  \to&\trace{A}=\trace{B}\\
  &\det{A}=\det{B}\\
  &{\lambda}_j (A)={\lambda}_j (B)
\end{align}

しかし,正則行列$S$をとってきても$\inverse{S}AS=B$とはこの場合できない$\to$Jordan標準形が$A$と$B$で違う.
\subsubsection{Jordan標準形}
Jordan細胞を以下のように$k\times k$行列で固有値$\lambda$の右横だけ成分が1の行列として定義する.
\begin{equation}
  J_k (\lambda )=\fbox{
  $\begin{array}{ccccc}
    \lambda&1&&&0\\
    &\lambda&1&&\\
    &&&\ddots&\\
    &&&&1\\
    0&&&&\lambda
  \end{array}$}
\end{equation}

$n\times n$行列$A$のJordan標準形は

\begin{equation}
  \inverse{S}AS=\left( \begin{array}{cccc}
    \elementinbox{J_{k_1}({\lambda}_1 )}&&&0\\
    &\elementinbox{J_{k_2}({\lambda}_2 )}&&\\
    &&\ddots&\\
    0&&&\elementinbox{J_{k_l}({\lambda}_l )}
  \end{array}\right)=J
\end{equation}
ここで$k_1 +k_2 +\cdots +k_l =n$であり${\lambda}_1 \cdots {\lambda}_l$は$A$の固有値
\begin{itemize}
  \item[(1)]$l$は線形独立な固有ベクトルの数\\
  \item[(2)]$A$が対角化可能$\Leftrightarrow l=n$\\
  \item[(3)]$\lambda$に対する$J_{k_i}({\lambda} )$の数=$\lambda$の幾何学的重複度\\
  \item[(4)]$\lambda$に対する$k_i$の和=$\lambda$の代数的重複度\\
  \item[(5)]$J_{k_i}({\lambda} )$のサイズは$J-\lambda\unitmatrix$のべき乗からわかる.
\end{itemize}
\subsubsection{Jordan標準形の性質の例}
\begin{equation}
  J=
  \begin{array}{|cccc|}
    \hline
    \begin{array}{|ccc|}
      \hline
      2&1&\\
      &2&1\\
      &&2\\\hline
    \end{array}&&&\\
    &
    \begin{array}{|cc|}
      \hline
      2&1\\
      &2\\\hline
    \end{array}&&\\
    &&\begin{array}{|cc|}
      \hline
      2&1\\
      &2\\\hline
    \end{array}&\\
    &&&2\\\hline
  \end{array}
\end{equation}

\begin{equation}
  J_k (0) =
  \left(
  \begin{array}{ccccc}
    0&1&&&\\
    &0&1&&\\
    &&&\ddots&\\
    &&&&1\\
    &&&&0
  \end{array}\right) は\mathrm{nilpotent}(ある正の整数nに対して{(J_k (0))}^n =0になること)
\end{equation}
\begin{align}
  &J-2\unitmatrix = \left(
  \begin{array}{cccc}
    J_3 (0)&&&\\
    &J_2 (0)&&\\
    &&J_2 (0)&\\
    &&&0
  \end{array}\right)=
  \left(\begin{array}{cccccccc}
    0&1&0&0&0&0&0&0\\
    0&0&1&0&0&0&0&0\\
    0&0&0&0&0&0&0&0\\
    0&0&0&0&1&0&0&0\\
    0&0&0&0&0&0&0&0\\
    0&0&0&0&0&0&1&0\\
    0&0&0&0&0&0&0&0\\
    0&0&0&0&0&0&0&0\\
  \end{array}\right)\\
  &{(J-2\unitmatrix)}^2 =
  \left(\begin{array}{cccccccc}
    0&0&1&0&0&0&0&0\\
    0&0&0&0&0&0&0&0\\
    0&0&0&0&0&0&0&0\\
    0&0&0&0&0&0&0&0\\
    0&0&0&0&0&0&0&0\\
    0&0&0&0&0&0&0&0\\
    0&0&0&0&0&0&0&0\\
    0&0&0&0&0&0&0&0\\
  \end{array}\right)\\
  &{(J-2\unitmatrix)}^3=0
\end{align}

Jordan標準形の性質を適用していくと
\begin{itemize}
  \item ${(J-2\unitmatrix)}^3 =0\to$最大ブロックの次数は3\\
  \item $\rank{{(J-2\unitmatrix)}^2}=1\to$次数3のブロックの数は1個\\
  \item $\rank{J-2\unitmatrix}=4\to 4=\underbrace{2}_{次数3のランク}\times\underbrace{1}_{次数3の個数}+\underbrace{1}_{次数2のランク}\times \underbrace{2}_{次数2の個数がわかる}$\\
  \item $J$の次数は8$\to 8=3\times 1+2\times 2+1\times \underbrace{1}_{次数1の個数がわかる}$
\end{itemize}
\subsubsection{Jordan標準形の利用}
常微分方程式の解法に利用できる.
\begin{equation}
  \begin{cases}
    \dot{x}(t)=Ax\\
    x(0)=x_0
  \end{cases}
\end{equation}

$A(n\times n)$のJordan標準形を$J$とすると正則行列$S$を用いて$A=SJ\inverse{S}$
\begin{equation}
  \begin{cases}
    \dot{y}(t)=Jy(t)\\
    y(0)=y_0
  \end{cases}\ \ \
  \begin{cases}
    x(t)=Sy(t)\\
    y_0 = \inverse{S}x_0
  \end{cases}
\end{equation}

$J$が対角行列なら以下のように簡単に解ける.
\begin{align}
  &\dot{y_l}(t)={\lambda}_i y_i (t)(i=1,2\cdots n)\\
  &y_i (t)=y_i (0)e^{{\lambda}_i t}
\end{align}

$J$が対角でない場合$J=J_n (\lambda )$だとする.

最後の行について$\dot{y_n}(t)=\lambda y_n (t)よりy_n (t)=y_n (0)e^{\lambda t}$となる.
\begin{align}
  &\dot{y_{n-1}}(t)=\lambda y_{n-1}(t)+y_n (0)e^{\lambda t}\\
  &y_{n-1}(t) = y_{n-1}(0)+y_n (0)t e^{\lambda t}
\end{align}
最後は特殊解求めるやつをつかう.
\section{グラフと行列}
$A=(a_{ij})(n\times n行列)$について
\begin{align}
  &V=\{ 1,2,\cdots n\}\\
  &E=\{ (i,j)| a_{ij}\neq 0\}
\end{align}

\begin{align}
  Gが強連結&\Leftrightarrow Aが既約\\
  &\Leftrightarrow PA\transpose{P}(P:置換行列)=
  \begin{array}{|c|c|}
    \hline
    &\\\hline
    0&\\\hline
  \end{array}\ にできない
\end{align}

2部グラフのマッチングとはいくつかの枝の集合で行同士列同士がぶつからないもののこと.$M:マッチング$としたとき$\max |M|$を最大マッチングという.

また,任意の枝$e$に対して$(eの左端点)\in X$または$(eの右端点)\in Y$となるとき$(X,Y)$を被覆(カバー)という.
\subsection{Koning-Egervaryの定理}
\begin{equation}
  \max |M|=\min (|X|+|Y|)
\end{equation}
ここで$M$はマッチングで$(X,Y)$はカバーである.

この定理からいえることとして
\begin{itemize}
  \item[(1)]任意のマッチング$M$と任意のカバー$(X,Y)$に対して$|M|\leq |X|+|Y|$\\
  \item[(2)]あるマッチングとあるカバーに対して$|M|=|X|+|Y|$
\end{itemize}

