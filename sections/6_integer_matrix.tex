\section{整数行列}
$A=(a_{ij})\ \ \ a_{ij}\in\mathbb{Z}$\ \ \ いま$Ax=b$を解くことを考える.
\begin{align}
  &{}^{\exists}x\in\realnspace{n}\Rightarrow \rank{[A|b]}=\rank{A}\\
  &{}^{\exists}x\in{\mathbb{Z}}^n \Rightarrow \text{答えは整数であるか?}
\end{align}
\subsection{単模行列(Unimodular行列)}
\begin{dfn}
  $Q:n\times n$整数行列が単模行列$\Leftrightarrow\det Q\in \{1,-1\}$
\end{dfn}
\begin{dfn}
  $Q:n\times n$整数行列が完全単模行列$\Rightarrow Q$は任意の正方部分行列$C$について$\det C\in {0,1,-1}$
\end{dfn}
\begin{itembox}[l]{
$Q:$整数行列の時以下はすべて同値.}
\begin{itemize}
  \item[(1)]$Q$が単模行列
  \item[(2)]$Q^{-1}$が整数行列
  \item[(3)]$Qx$が整数ベクトル$ \Leftrightarrow x$が整数ベクトル
  \item[(4)]$Q$が整数行列に対する列基本変形を表す行列の積
  \item[(5)]$Q$が整数行列に対する行基本変形を表す行列の積
\end{itemize}
\end{itembox}
\begin{itemize}
  \item[(1)$\to$(2)]$Q^{-1}=\displaystyle\frac{余因子行列}{\det Q}$\\
  \item[(2)$\to$(1)]$QQ^{-1}=I$の行列式を考えて$\det Q\det Q^{-1}=1$\ \ \ いま$Q^{-1}$が整数行列なので$\det Q^{-1}\in\mathbb{Z}$から$\det Q=1,-1$\\
  \item[(2)$\to$(3)]$Qx\in{\mathbb{Z}}^n $について$y=Qx$とおくと$x=Q^{-1}y\in{\mathbb{Z}}^n$
\end{itemize}

$\realnspace{1}$の場合の列基本変形は
\begin{itemize}
  \item ある列を$\alpha$倍
  \item 2つの列を入れ替える
  \item ある列$\times\alpha$を他の列に加える
\end{itemize}

\hypertarget{basictrans}{$\mathbb{Z}$}の場合は
\begin{itemize}
  \item ある列を-1倍
  \item 2つの列を入れ替える
  \item ある列の整数倍を他の列に加える
\end{itemize}
(4)(5)については後ほど
\subsection{Hermite標準形}
与えられた整数行列を整数基本列変形によって
できるだけ簡単な形に変換することを考える.
Hermite標準形は一意.
\begin{itembox}[l]{Hermite標準形}
$A:(m\times n)$整数行列,$\rank A = m$.
ある単模行列$Q$が存在して
\begin{equation}
  AQ=
  \begin{pmatrix}
    \begin{array}{ccc|}
      {\beta}_{11}&&0\\
      \vdots&\ddots&\\
      {\beta}_{m1}&&{\beta}_{mn}
    \end{array}0
  \end{pmatrix}
\end{equation}
\begin{enumerate}
    \item 下三角${\beta}_{ij}=0(i<j)$
    \item 非負${\beta}_{ij}\geq 0,{\beta}_{ii}>0$
    \item 行方向に見たときに対角項が最大${\beta}_{ii}>{\beta}_{ij}(i>j)$
\end{enumerate}
つまり
非負の要素からなる左下三角行列で,各行にお
いて.対角要素が非対角要素よりも大きいような整数行列である.
列基本変形を繰り返すことで,
Hermite標準形をつくることができる.
\end{itembox}
\begin{thm}
  任意の単模行列は列基本変形の積
\begin{proof}
単位行列$I$に列基本変形を施して$A$になるならば$A$は単模行列(列基本変形は$\det$を符号以外変えない).
\end{proof}
\end{thm}
また実は単模行列に列基本変形を施せばHermite標準形になることを示す.

一つの行に着目する: $
\begin{array}{|c|}
  \hline
  a_1 a_2 \cdots a_n\\\hline
\end{array}
$,$a_i$はスカラー.
\begin{itemize}
  \item[1]いくつかの列を-1倍して$a_1 \cdots a_n \geq 0$とする.
  \item[2]列を入れ替えて$a_1 \geq a_2 \geq \cdots a_n \geq 0$とする.\\
  このとき$a_1 >0$である,なぜなら$a_1 =0$だとすべての行が0で行フルランクに反してしまうから.
  \item[3]列を互いに引き算する.$a_i = a_j q_i +r_i (a_i \geq a_j でありq_i \in\mathbb{Z}>0でr_i は余り)$を計算し$a_i =r_i$にして2に戻ることを繰り返す.
\end{itemize}

もし$a_2 \neq 0$ならば$a_1 -a_2$が可能でありいつかは$a_2 =0$になる.それを他の行でも同じことをする.

\subsection{\texorpdfstring{$Ax=b$の$\mathbb{Z}$} 上での可解性}
$\mathbb{R}$ならば
\begin{equation}
    \mathrm{rank}[A|b] = \mathrm{rank} A
\end{equation}
\begin{itembox}[l]{$\mathbb{Z}$上での可解性}
  $A:m\times n$整数行列で$\rank{A}=m$とすると以下はすべて同値である.
\begin{enumerate}
  \item 任意の$m$次整数ベクトル$b$に対して$Ax=b$が整数ベクトルの解$x$をもつ.
  \item 実数ベクトル$y$について$\transpose{y}A\in{\mathbb{Z}}^n \Rightarrow y\in{\mathbb{Z}}^m$
  \item $g_m (A)=1$ ($A$の行列式因子$g_k (A)\cdots k$次の小行列式の最大公約数)
  
  \item $A$のHermite標準形=$[I_m |0]$
\end{enumerate}
\end{itembox}
(4)を基準に見ていく.

まず(1)について.単模行列$Q$
を用いて,
$AQ$をHermite標準形にして
\begin{equation}
    Ax=b\Leftrightarrow AQ\inverse{Q}x=b
\end{equation}

先週より$x\in{\mathbb{Z}}^n \Leftrightarrow\inverse{Q}x\in{\mathbb{Z}}^n (4\to 1)$なので$A$をHermite標準形としてよい.

${}^{\forall}b\in{\mathbb{Z}}^m$に対して
\begin{equation}
  \begin{tabular}{|c|c|}
    \hline
    \backslashbox{$I$}{0}&0\\
    &\\\hline
  \end{tabular}\
  \begin{array}{|c|}
    \hline
    \\
    x\\
    \\\hline
  \end{array}=
  \begin{array}{|c|}
    \hline
    \\
    b\\
    \\\hline
  \end{array}
\end{equation}
このとき$i(>1)$行目で
\begin{equation}
  \begin{tabular}{|cc|c|}
    \hline
    \backslashbox{$I$}{0}&&0\\\hline
    &\backslashbox{a}{}&\\\hline
    &&\\\hline
  \end{tabular}\
  \begin{array}{|c|}
    \hline
    \\
    x\\
    \\\hline
  \end{array}=
  \begin{array}{|c|}
    \hline
    0\\
    \vdots\\
    b_i\\
    \vdots\\
    0
    \\\hline
  \end{array}
\end{equation}
$1~i-1$行は解けて$x=
\begin{array}{|c|}
  \hline
  0\\
  \vdots\\
  x_i\\
  \vdots\\
  \ast
  \\\hline
\end{array}\to ax_i =b_i $.任意にとれる$b_i\in\mathbb{Z}$なので$a=1$
\\
次に(2)についてみてみる.$\transpose{y}A\in{\mathbb{Z}}^n \Leftrightarrow\transpose{y}AQ\in{\mathbb{Z}}^n$($Q$は単模行列)なので$A$をHermite標準形としてよい.

このとき(2)$\Leftrightarrow{\beta}_{11}={\beta}_{22}=\cdots{\beta}_{mm}=1$を示す.
\begin{equation}
\begin{pmatrix}
    \beta_{11} &0&& \cdots& 0\\
    \beta_{12} &\beta_{22}& 0& \cdots& 0\\
    \cdots & && &0
    \end{pmatrix}
\end{equation}
$\Leftarrow$は明らかである.
\begin{equation}
    A = [I_m|0] \quad y^T A = [y^T | 0]
\end{equation}
$\Rightarrow{\beta}_{ii}\neq 1$となる最初の行($k$行)に注目する.

\begin{align*}
  &A=
  \begin{tabular}{|cc|c|}
    \hline
    \backslashbox{$I$}{0}&&0\\\hline
    &\backslashbox{${\beta}_{k1}{\beta}_{k2}\cdots$ ${\beta}_{kk}$}{}&\\\hline
    &&\\\hline
  \end{tabular}\\
  &\transpose{y}=
  \begin{array}{|c|c|c|c|c|}
    \hline
    -{\beta}_{k1}/{\beta}_{kk}&-{\beta}_{k2}/{\beta}_{kk}&\cdots&1/{\beta}_{kk}&0\\\hline
  \end{array}
\end{align*}
とおくと$\transpose{y}A=[0,0\cdots 1,0\cdots 0]$となって$\transpose{y}A\in{\mathbb{Z}}^n$であるが$y\notin{\mathbb{Z}}^n$
よって,$\beta_{ii} = 1$\\
次に(3)について,
\begin{itembox}[l]{準備}
$g_k (A)$は$A$の$\mathbb{Z}$型の
\hyperlink{basictrans}{列基本変形}で不変である.
\begin{enumerate}
    \item 列$i$を-1倍
    \item 列$i$と列$j$を入れ替える
    \item 列$j$の整数倍を列$i$に加える
\end{enumerate}
$A$に対して$\alpha=1~k$列の小行列式で$\beta=1~k-1,j$列の小行列式
とする.$A$の$j$列を$i$列に加えて$A'$とする.
$A'$に対して$1~k-1,j$列の小行列式 $=\beta' = \beta $,$1~k$列の小行列式を$a'$とする.\\
$A'$の$1~k$列の小行列式のLaplace展開を考えると
\begin{align}
   &\text{ ($a_i$に関する展開)+($a_j$に関する展開)}\\
   & = \alpha + \beta\\
   & = \alpha'
\end{align}
\begin{equation}
    \mathrm{gcd}(\alpha', \beta)
    = \mathrm{gcd} (\alpha + \beta, \beta)
    = \mathrm{gcd} (\alpha, \beta)
\end{equation}
\end{itembox}
準備より$Q$が単模行列なら
\begin{equation}
g_m (A)=g_m (AQ)
\end{equation}
なので$A$をHermite標準形としてよい.$g_m (A)=1\Leftrightarrow{\beta}_{11}=\cdots ={\beta}_{mm}=1$
\subsection{Smith標準形}
Hermite標準形は列変形だけによる三角化であるが,
列変形と行変形の両方を用いると対角行列にする
ことができる.Smith標準形は一意に定まる.
\begin{itembox}[l]{Smith標準形}
$A\in{\mathbb{Z}}^{m\times n},\ \rank{A}=r,\ {}^{\exists}P,Q$を単模行列とする.
\begin{equation}
  PAQ=
  \begin{array}{|c|c|}
    \hline
    \begin{array}{ccc}
      {\alpha}_1&&\\
      &\ddots&\\
      &&{\alpha}_r
    \end{array}&0\\\hline
    0&0\\\hline
  \end{array}
\end{equation}
$\alpha_1\leq \cdots \leq \alpha_r(r =\mathrm{rank} A)$は
正の整数であり,$\alpha_k$が$\alpha_{k+1}$を割り切る.
$\alpha_k$は$A$の単因子と呼ばる.
このとき
\begin{equation}
    g_k (A)=g_k (PAQ)={\alpha}_1 \cdots{\alpha}_k
\end{equation}
\end{itembox}
例えば$k=1$だと,
小行列式は$\alpha_1,\cdots, \alpha_r$となるので
\begin{equation}
    g_1 (A) = \alpha_1
\end{equation}
Smith標準形の作り方を述べる.
(整除条件を無視すると)
まず$\min\{ |a_{ij}|:i=1\cdots m,\ j=1\cdots n\, a_{ij}\neq 0\}$なる$(i,j)$に注目して$a_{ij}$を基本変形で(1,1)にもってくる.
次に列基本変形で$0\leq a_{1j}<a_{11} (j=2\cdots n)$とする.$a_{ij}=q_j a_{11}+r_j (q_j \in\mathbb{Z})$として${a_{ij}}^{'}=r_j$とする.$a_{ij}$を$a_{11}$で割り算して余りで更新していく.これを続けて最小($\neq 0$)になった$a_{ij}$を(1,1)にもってくる.そうするといつかは以下のようになる.
\begin{equation}
  \begin{array}{|cccc}
    \hline
    a_{11}&0&\cdots&0\\\hline
    &&&\\
    &&&
  \end{array}
\end{equation}

同じことを行基本変形でも行うと以下のようになる.
\begin{equation}
  \begin{array}{c|ccc}
    a_{11}&0&\cdots&0\\\hline
    0&&&\\
    \vdots&&a_{11}の倍数&\\
    0&&&
  \end{array}
\end{equation}
もし$a_{11}$の倍数でないなら$a_{ij}\to a_{ij}/a_{11}$の剰余ができるので.
$a_{11}$の最小性に反する.
(整除条件)について
$\alpha_{i} | \alpha_{i+1}$を満たさない最小の$i$について,
行番号と列番号が
{$i,i+1$}の部分の
diag($\alpha_i$,$\alpha_{i+1}$)に注目する.
\begin{equation}
    \alpha_i v + \alpha_{i+1} u =\mathrm{gcd}(\alpha_i,\alpha_{i+1})
\end{equation}
を満たす整数$u$,$v$をとり
\begin{equation}
    \begin{bmatrix}
    1 & u\\
    0 & 1
    \end{bmatrix}
    \begin{bmatrix}
    \alpha_i &0\\
    0 & \alpha_{i+1}
    \end{bmatrix}
    \begin{bmatrix}
    1 & v\\
    0 & 1
    \end{bmatrix}
    = 
    \begin{bmatrix}
    \alpha_i & \mathrm{gcd}(\alpha_i,\alpha_{i+1})\\
    0 & \alpha_{i+1}
    \end{bmatrix}
\end{equation}
列を入れ替えてから
\begin{equation}
    \begin{bmatrix}
    1 & 0\\
    -\alpha_{i+1}/\mathrm{gcd}(\alpha_i,\alpha_{i+1}) & 1
    \end{bmatrix}
    \begin{bmatrix}
    \mathrm{gcd}(\alpha_i,\alpha_{i+1}) & \alpha\\
    \alpha_{i+1} & 0
    \end{bmatrix}
    \begin{bmatrix}
    1 & -\alpha_i/\mathrm{gcd}(\alpha_i,\alpha_{i+1}) \\
    0 & 1
    \end{bmatrix}
    =
    \begin{bmatrix}
      \mathrm{gcd}(\alpha_i,\alpha_{i+1})  & 0\\
    0 & -\alpha_i \alpha_{i+1}/  \mathrm{gcd}(\alpha_i,\alpha_{i+1}) 
    \end{bmatrix}
\end{equation}