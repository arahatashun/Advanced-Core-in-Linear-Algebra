$Ax=b$の$\mathbb{Z}$上での可解性について考える.
\begin{thm}
  $A:m\times n$整数行列で$\rank{A}=m$とすると以下はすべて同値である.
\end{thm}
\begin{itemize}
  \item[(1)]${}^{\forall}b\in{\mathbb{Z}}^m$に対して${}^{\exists}x\in{\mathbb{Z}}^n :Ax=b$\\
  \item[(2)]$\transpose{y}A\in{\mathbb{Z}}^n \Rightarrow y\in{\mathbb{Z}}^m$\\
  \item[(3)]$A$の行列式因子$g_k (A)\cdots k$次の小行列式の最大公約数として,$g_m (A)=1$\\
  \item[(4)]$A$のエルミート標準形=$[I_m |0]$
\end{itemize}

(4)を基準に見ていく.

まず(1)について.$Q$を単摸行列でエルミート標準形として$Ax=b\Leftrightarrow AQ\inverse{Q}x=b$

先週より$x\in{\mathbb{Z}}^n \Leftrightarrow\inverse{Q}x\in{\mathbb{Z}}^n (4\to 1)$なので$A$をエルミート標準形としてよい.

${}^{\forall}b\in{\mathbb{Z}}^m$に対して
\begin{equation}
  \begin{tabular}{|c|c|}
    \hline
    \backslashbox{$I$}{0}&0\\
    &\\\hline
  \end{tabular}\
  \begin{array}{|c|}
    \hline
    \\
    x\\
    \\\hline
  \end{array}=
  \begin{array}{|c|}
    \hline
    \\
    b\\
    \\\hline
  \end{array}
\end{equation}
このとき$i(>1)$行目で
\begin{equation}
  \begin{tabular}{|cc|c|}
    \hline
    \backslashbox{$I$}{0}&&0\\\hline
    &\backslashbox{a}{}&\\\hline
    &&\\\hline
  \end{tabular}\
  \begin{array}{|c|}
    \hline
    \\
    x\\
    \\\hline
  \end{array}=
  \begin{array}{|c|}
    \hline
    0\\
    \vdots\\
    b_i\\
    \vdots\\
    0
    \\\hline
  \end{array}
\end{equation}
$1~i-1$行は解けて$x=
\begin{array}{|c|}
  \hline
  0\\
  \vdots\\
  x_i\\
  \vdots\\
  \ast
  \\\hline
\end{array}\to ax_i =b_i \in\mathbb{Z}$なので$a=1$

次に(2)についてみてみる.$\transpose{y}A\in{\mathbb{Z}}^n \Leftrightarrow\transpose{y}AQ\in{\mathbb{Z}}^n$($Q$は単摸行列)なので$A$をエルミート標準形としてよい.

このとき(2)$\Leftrightarrow{\beta}_{11}={\beta}_{22}=\cdots{\beta}_{mm}=1$を示す.

$\Leftarrow$は明らかである.

$\Rightarrow{\beta}_{ii}\neq 1$となる最初の行($k$行)に注目する.

\begin{align*}
  &A=
  \begin{tabular}{|cc|c|}
    \hline
    \backslashbox{$I$}{0}&&0\\\hline
    ${\beta}_{k1}{\beta}_{k2}\cdots$&\backslashbox{${\beta}_{kk}$}{}&\\\hline
    &&\\\hline
  \end{tabular}\\
  &\transpose{y}=
  \begin{array}{|c|c|c|c|c|}
    \hline
    -{\beta}_{k1}/{\beta}_{kk}&-{\beta}_{k2}/{\beta}_{kk}&\cdots&1/{\beta}_{kk}&0\\\hline
  \end{array}
\end{align*}
とおくと$\transpose{y}A=[0,0\cdots 1,0\cdots 0]$となって$\transpose{y}A\in{\mathbb{Z}}^n$であるが$y\notin{\mathbb{Z}}^n$

次に(3)について.

準備として$g_k (A)$は$A$の$\mathbb{Z}$型の\hyperlink{basictrans}{列基本変形}で不変である.

$A$に対して$\alpha=1~k$列の小行列式で$\beta=1~k-1,j$列の小行列式とし$A$の$j$列を$i$列に加えて$A'$とする.

$A'$に対して$1~k-1,j$列の小行列式$=b$,$1~k$列の小行列式を$a'$とする.Laplace展開を考えると($a_i$に関する展開)+($a_j$に関する展開)なので$a'=a+b$

準備より$Q$が単摸なら$g_n (A)=g_m (AQ)$なので$A$をエルミート標準形としてよい.$g_m (A)=1\Leftrightarrow{\beta}_{11}=\cdots ={\beta}_{mm}=1$
\subsection{Smith標準形}
$A\in{\mathbb{Z}}^{m\times n},\ \rank{A}=r,\ {}^{\exists}P,Q$を単摸行列とする.
\begin{equation}
  PAQ=
  \begin{array}{|c|c|}
    \hline
    \begin{array}{ccc}
      {\alpha}_1&&\\
      &\ddots&\\
      &&{\alpha}_r
    \end{array}&0\\\hline
    0&0\\\hline
  \end{array}
\end{equation}

${\alpha}_1 >0,{\alpha}_2 >0\cdots {\alpha}_r >0$であり${\alpha}_1 |{\alpha}_2 \cdots |{\alpha}_r ({\alpha}_1 は{\alpha}_2 を割り切る)$

このとき$g_k (A)=g_k (PAQ)={\alpha}_1 \cdots{\alpha}_k$

Smith標準形の作り方を述べる.まず$\min\{ |a_{ij}|i=1\cdots m,\ j=1\cdots n\ a_{ij}\neq 0\}$なる$(i,j)$に注目して$a_{ij}$を基本変形で(1,1)にもってくる.

次に列基本変形で$0\leq a_{1j}<a_{11} (j=2\cdots n)$とする.$a_{ij}=q_j a_{11}+r_j (q_j \in\mathbb{Z})$として${a_{ij}}^{'}=r_j$とする.$a_{ij}$を$a_{11}$で割り算して余りで更新していく.これを続けて最小($\neq 0$)になった$a_{ij}$を(1,1)にもってくる.そうするといつかは以下のようになる.
\begin{equation}
  \begin{array}{|cccc}
    \hline
    a_{11}&0&\cdots&0\\\hline
    &&&\\
    &&&
  \end{array}
\end{equation}

同じことを行基本変形でも行うと以下のようになる.
\begin{equation}
  \begin{array}{c|ccc}
    a_{11}&0&\cdots&0\\\hline
    0&&&\\
    \vdots&&a_{11}の倍数&\\
    0&&&
  \end{array}
\end{equation}
もし$a_{11}$の倍数でないなら$a_{ij}\to a_{ij}/a_{11}$の基本変形ができるので.
