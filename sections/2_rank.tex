\section{階数(ランク)}
\subsection{ランクの定義}
$A(m\times n行列)$について
\begin{itemize}
  \item[1] 部分行列が正則であるような最大のサイズ\\
  \item[2] 1次独立な行,列ベクトルの最大の数\\
  \item[3] $A:\spacemap{n}{m}$として像空間の次元$\dim (\mathrm{Im}A)$\\
  \item[4] $A:\spacemap{n}{m}$として核の余次元
\end{itemize}
ランクはトラス構造の安定性やグラフの剛性の解析に使う.
\subsection{ランクの性質}
\begin{itemize}
  \item[1] $\mathrm{rank}(AB)\leq \min\{ \mathrm{rank}(A),\mathrm{rank}(B)\}$\\
  \item[2] $S,T$が正則行列だとすると$\mathrm{rank}(A)=\mathrm{rank}(SAT)$\\
  \item[3] $\mathrm{rank}([A:B])\leq \mathrm{rank}(A)+\mathrm{rank}(B)$転置をとって縦に連結しても同様.\\
  \item[4] $\mathrm{rank}\left(
  \begin{bmatrix}
  A & B\\
  C & 0
  \end{bmatrix}
  \right)\geq \mathrm{rank}(B)+\mathrm{rank}(C)$\\
  \item[5] $\mathrm{rank}(A+B)\leq \mathrm{rank}(A)+\mathrm{rank}(B)$\\
  \item[6] \textcolor{red}{Frobenius不等式} $\mathrm{rank}(ABC)\geq \mathrm{rank}(AB)+\mathrm{rank}(BC)-\mathrm{rank}(B)$
\end{itemize}
\subsection{ランクの性質の証明}
1の証明について.

Binet-Cauchyの定理$\det (AB)=\displaystyle\sum_{J}\det A[J]\det B[J]$

2の証明について.

$\mathrm{rank}(SAT)\leq \mathrm{rank}(A)$かつ$\mathrm{rank}(A)=\mathrm{rank}(\inverse{S}(SAT)\inverse{T})\leq \mathrm{rank}(SAT)$より

3の証明について.

$\mathrm{rank}([A:B])$について$A,B$のなかに従属している縦ベクトルがあるかもしれないので.

4の証明について.

4の証明について.

$\fourmatrix{A}{I}{B}{-I}\xrightarrow[1行に加える]{第2行を}\fourmatrix{A+B}{0}{B}{-I}\xrightarrow[1列に加える]{第2列をB倍して}\fourmatrix{A+B}{0}{0}{-I}$

$\fourmatrix{A}{I}{B}{-I}$のrank$\leq \mathrm{rank}(A+B)+m\leq \mathrm{rank}(A)+\mathrm{rank}(B)+m$

6の証明について.

$\fourmatrix{B}{BC}{0}{ABC}\xrightarrow{2列-1列\times C}\fourmatrix{B}{0}{0}{ABC}$であり右辺のランクは$\mathrm{rank}(B)+\mathrm{rank}(ABC)$

一方で$\fourmatrix{B}{BC}{0}{ABC}\xrightarrow{2行-1行\times A}\fourmatrix{B}{BC}{-AB}{0}\geq \mathrm{rank}(AB)+\mathrm{rank}(BC)(4の性質より)$
