\section{非負行列}
\begin{dfn}
  $A$が非負行列$\Leftrightarrow$要素ごとに$a_{ij}\geq 0であり,このときA\geq 0$とかく
\end{dfn}
\begin{dfn}
  $A$が正行列$\Leftrightarrow a_{ij}>0$
\end{dfn}

たとえばマルコフ連鎖を考えると推移行列は非負である.
\begin{align}
  &p_{ij}\geq 0({}^{\forall}i,j)\\
  &\displaystyle\sum_{j=1}^n p_{ij}=1 ({}^{\forall}i)
\end{align}
\begin{dfn}
  $A$が確率行列$\Leftrightarrow A\geq 0$かつ
  $\sum_{j=1}^n a_{ij}=1({}^{\forall}i)$(各行の和が1)
\end{dfn}
\begin{dfn}
  $A$が二重確率行列$\Leftrightarrow Aも\transpose{A}も確率行列$
\end{dfn}
\subsection{Birkhoff-von Neumannの定理}
\begin{itembox}[l]{Birkhoffの定理}
  任意の二重確率行列は置換行列の凸結合(重み$\geq 0$,重みの和$=1$)で表せる.
\end{itembox}
置換行列とは各行各列にちょうど一個だけ1を持ち,残りはすべて0の行列であり,完全マッチングに対応している.
\begin{proof}
$A$の非ゼロ要素数についての帰納法で行う.グラフの枝の数を$v$とする.\\
(1) $v=n$のとき$A$自身が置換行列である.\\
(2) 一般に
$|\Gamma (X)|\geq |X|({}^{\forall}x\subseteq U)$
のとき完全マッチングが存在する.
\begin{align}
  |\Gamma (X)| &=\sum_{j\in \Gamma (X)}1\\
  &=\sum_{j\in\Gamma (X)}\left(\sum_{i=1}^n a_{ij}\right)
  (列和が1なので)\\
  &\geq \sum_{j\in\Gamma (X)} \left(\sum_{i\in X}a_{ij}\right)\\
  &=\sum_{i\in X}\sum_{j\in\Gamma (X)}a_{ij}\\
  &=\sum_{i\in X}\sum_{j=1}^n a_{ij}\\
  &=\sum_{i\in X}1(行和が1なので)\\
  &=|X|
\end{align}
$M:$完全マッチングを選ぶと置換行列$P_M$が対応する.
$\mu :\min_{(i,j)\in M}a_{ij}(a_{ij} > 0)$とする.

$\tilde{A}=A-\mu P_M$ とおくと$\tilde{A}$は非負行列でありかつ
行和,
列和$=1-\mu$である.さらに,
$(\tilde{A}の非ゼロ数)\leq (Aの非ゼロ数)-1$であり以下のように$\tilde{A}$が凸結合で表せると仮定する.すると$A$も凸結合で表せる.
\begin{align}
  \displaystyle\frac{1}{1-\mu}\tilde{A}&=\sum_{i}{\alpha}_i P_i (凸結合)\\
  A&=\tilde{A}+\mu P_M\\
  &=\displaystyle\sum_i (1-\mu ){\alpha}_i P_i +\mu P_M
\end{align}
\end{proof}
\subsection{既約な行列の定理}
\begin{itembox}[l]{既約な行列の定理}
\begin{enumerate}
\item $A$が既約$\Leftrightarrow$ 置換行列$P$を用いて
\begin{equation}
    P^TAP =   \begin{array}{|c|c|}
    \hline
    &\\\hline
    0&\\\hline
  \end{array}\ にできない
\end{equation}
とできない
\item $A$を$n\times n$行列として,
$A\geq 0$で$A$が既約なら${(I+A)}^{n-1}>0$(正行列)

\item (別な言い方)既約な$n$次非負行列$A$に対して,
$m \geq n-1$ならば
\begin{equation}
    (I+A)^{m} > 0
\end{equation}
が成り立つ.
\end{enumerate}
\end{itembox}
\begin{proof}
${(I+A)}^{n-1}x>0\ {}^{\forall}x\geq 0(x\neq 0)$を示せばよい.
ベクトル列$\{x_k\}$を
\begin{equation}
x_{k+1}=(I+A)x_k\quad (x_0 \equiv x)    
\end{equation}
とすると$x_k \to x_{k+1}でゼロ要素数が減ることを示す.$
\begin{align}
  &x_k =
  \begin{array}{|c|}
     \hline
     \alpha\\\hline
     0\\\hline
  \end{array}(\alpha >0)\ \ \ A=
  \begin{array}{|c|c|}
    \hline
    A_{11} & A_{12}\\\hline
    A_{21} & A_{22}\\\hline
  \end{array}\\
  &x_{k+1}=
  \begin{array}{|c|}
     \hline
     \beta\\\hline
     \gamma\\\hline
  \end{array}=
    \begin{array}{|c|}
    \hline
    \alpha\\\hline
    0\\\hline
  \end{array}
  +
  \begin{array}{|c|}
    \hline
    A_{11} \alpha\\\hline
    A_{21} \alpha\\\hline
  \end{array}
\end{align}

$A_{11}\geq 0,\alpha >0より\beta >0$\\
$A$は既約なので$A_{21}\neq 0,\alpha >0よりA_{21}\alpha \neq 0$なので($x_{k+1}$のゼロの数)$<$($x_k$のゼロの数)
\end{proof}
\subsection{Perron-Frobeniusの定理}
$A\geq 0$で既約としスペクトル半径は$\rho (A)=\max |\lambda |(\lambda :固有値)$である.
\begin{itembox}[l]{Perron-Frobeniusの定理(1)}
既約な非負行列$A$は,次の性質をもつ
固有値$\alpha$を持つ.
  \begin{itemize}
    \item $\alpha$に対応する固有ベクトルとして,
    要素がすべて正のベクトルを取れる.
    \item $\alpha$は単純固有値(固有方程式の単純根).
    \item $A$のその他の固有値の絶対値は$\alpha$を超えない.
  \end{itemize}
  (別の言い方)
  既約な非負行列$A$のスペクトル半径を$\rho(A)$とすると,
  $\rho(A) >0$であり,$\rho(A)$は$A$の単純固有値であって,
  対応する固有ベクトルは正ベクトルのスカラー倍である.
\end{itembox}
\begin{itembox}[l]{既約な場合の固有ベクトル}
既約な非負行列$A$は,正の固有ベクトル$u$をもつ.
このとき対応する固有値$\alpha$は正である.
(順番に注意)
\end{itembox}
途中の証明にGershgorinの定理を用いて,
確率行列の任意の固有値が1以下であることを用いる.
復習すると
\begin{itembox}[l]{Gershgorinの定理}
$n$次複素行列$A=(a_{ij})$のすべての固有値は,
各$i$に対して,対角要素$a_{ii}$を中心として,
その行の非対角要素の絶対値の和を半径とする
円板
\begin{equation}
    C_i = \left\{\lambda \in \mathbb{C}| |\lambda - a_{ii}| \leq 
    \sum_{j\neq i} |a_{ij}|
    \right \}
\end{equation}
$n$個の円板$C_1,C_2,\cdots$の和集合に含まれる.
より詳しくは,$C_1 \cup C_2 \cdots$の
各々の連結成分は,それを構成する円板の
個数と同数の固有値を含む.つまり,
$k$個の円板が一つの連結成分を構成する時,
$\lambda_j \in C_{i_1}\cup C_{i_2} \cup \cdots \cup
C_{i_k}$を満たす$j$の個数が$k$に等しい.
\end{itembox}
\subsection{周期}
行列$A$をグラフ$G$に,既約を強連結に対応させる.

\begin{itembox}[l]{周期}
  周期$\sigma$もしくは原始指数は
  すべての閉路の長さの最大公約数である.
  特に$\sigma$=1のとき,行列$A$は原始的という.
\end{itembox}
例えば,\\
$A=\fourmatrix{0}{1}{0}{1}$の場合閉路の長さは$1\to 2\to 1$で2であり$\sigma =2$\\
$A=\fourmatrix{0}{1}{1}{1}$の場合は閉路が$1\to 2\to 1と2\to 2$で$\sigma =1$
\\
\begin{itembox}[l]{有向道の長さ}
  各点$u,v \in V$に対して,
  $u$から$v$に至る有向道の長さは
  mod $\sigma$で一つに確定する.
\end{itembox}
\begin{proof}
$P_1,P_2$を点$u$から$v$への有向道として,
その長さを$l(P_1),l(P_2)$とする.
$v$から$u$への有向道が存在して,$Q$とすると,
有効閉路の長さは$\sigma$の倍数であるから.
\begin{align}
    l(P_1) + l(Q) &\equiv 0 \quad (\mathrm{mod} \sigma)\\
     l(P_2) + l(Q) &\equiv 0 \quad (\mathrm{mod} \sigma)\\
     \therefore l(P_1) &\equiv l(P_2) \quad (\mathrm{mod} \sigma)
\end{align}
\end{proof}
グラフで始点1を固定して考える.$k=1,\cdots,\sigma - 1$として
\begin{equation}
    I_k = \left\{ i\in V: \text{点1から$i$は長さ$l$}\equiv k(\mod\sigma)
    \text{で到達可能}\right\}
\end{equation}
以下ではA既約,非負,$\sigma =3$とする.
\begin{equation}
  \transpose{P}AP=
  \begin{array}{rccc}
    &\overset{I_0}{\leftrightarrow}&\overset{I_1}{\leftrightarrow}&\overset{I_2}{\leftrightarrow}\\
    I_0 \updownarrow&&&\\
    I_1 \updownarrow&&&\\
    I_2 \updownarrow&&&\\
  \end{array}=
  \begin{pmatrix}
    0&A_1 &0\\
    0&0&A_2\\
    A_3 &0&0
  \end{pmatrix}
\end{equation}
Perron-Frobeniusより
$\rho(A)$は$A$の固有値.\\
$|\lambda| = \rho(A)$となる固有値はどれぐらいあるのか?
\begin{align*}
  Ax&=\rho (A)x\\
  \begin{pmatrix}
    0&A_1 &0\\
    0&0&A_2\\
    A_3 &0&0
  \end{pmatrix}
  \begin{pmatrix}
    x_1\\
    x_2\\
    x_3
  \end{pmatrix}
  &=\rho (x)
  \begin{pmatrix}
    x_1\\
    x_2\\
    x_3
  \end{pmatrix}\\
  \Leftrightarrow
  \begin{cases}
    A_1 x_2 =\rho (A)x_1\\
    A_2 x_3 =\rho (A)x_2\\
    A_3 x_1 =\rho (A)x_3
  \end{cases}
\end{align*}

$\zeta = e^{\frac{2}{3}\pi i}$として
\begin{equation*}
  x^{(1)}=
  \begin{pmatrix}
    x_1 \\
    \zeta x_2\\
    {\zeta}^2 x_3
  \end{pmatrix},\ \ \
  x^{(2)}=
  \begin{pmatrix}
    x_1\\
    {\zeta}^2 x_2\\
    {\zeta}^4 x_3
  \end{pmatrix}とおくと
\end{equation*}
\begin{align}
  Ax^{(1)}&=
  \begin{pmatrix}
    0&A_1 &0\\
    0&0&A_2\\
    A_3 &0&0
  \end{pmatrix}
  \begin{pmatrix}
    x_1\\
    \zeta x_2\\
    {\zeta}^2 x_3
  \end{pmatrix}=
  \begin{pmatrix}
    A_1 \zeta x_2\\
    A_2 {\zeta}^2 x_3\\
    A_3 x_1
  \end{pmatrix}=
  \begin{pmatrix}
    \zeta \rho (A)x_1\\
    {\zeta}^2 \rho (A)x_2\\
    {\zeta}^3 \rho (A)x_3
  \end{pmatrix}=\zeta \rho (A)x^{(1)}\\
  Ax^{(2)}&=・・・{\zeta}^2 \rho (A)x^{(2)}
\end{align}

したがって${\zeta}^k \rho (A)\ (k=0,1,\cdots n-1)$は$A$の固有値である.
ほかにないことは,......伊理「線形代数汎論」p.280\\
$\lambda :A$の固有値,
$x_1$は$\lambda$に対応する固有ベクトルとして,
\begin{equation*}
  (A_1 A_2 A_3 )x_1 = A_1 A_2 (\lambda x_3 )=\cdots {\lambda}^3 x_1
\end{equation*}
より周期は1で実はスペクトル半径円上の固有値は$\rho (A_1 A_2 A_3 )$のみで
といえれば,
\begin{equation}
    \lambda^3 = \rho(A^3) = \rho(A)^3
\end{equation}
のみとなりOK.
\\
\begin{itembox}[l]{スペクトル円上の固有値}
  \begin{align}
  &A\geq 0,\ 既約.a_{ii}>0 \ ({}^{\forall}i)\text{ならば}\\
  &\lambda\neq\rho (A)がAの固有値\Leftrightarrow |\lambda |<\rho (A)\\
 &\text{つまりスペクトル円上のAの固有値は$\rho(A)$のみ}
\end{align}
\end{itembox}
\begin{proof}
仮定からPerron Frobeniusの定理を用いることができて,
$Au=\rho (A)\bm{u}\ \ \bm{u}>0$とすると,
\begin{equation}
  U=
  \begin{pmatrix}
    u_1&&&\\
    &u_2&&\\
    &&\ddots&\\
    &&&u_n
  \end{pmatrix}\quad \hat{A}=U^{-1}AUとする.
  \end{equation}
\begin{itemize}
  \item[1]$A$の固有値$\lambda$は$\hat{A}$の固有値\\
  $\because$相似変換,固有方程式を考えると
  \begin{align}
      |U^{-1}AU - \lambda I|
      =|U^{-1}AU - \lambda U^{-1}U|
      =|U^{-1}(A -\lambda I ) U|
      = |U^{-1}| |A-\lambda I | |U|
      = | A-\lambda I|
  \end{align}
  \item[2]$\sum_{j=1}^n
  \hat{a_{ij}}=\rho (A)$となる.なぜなら$ \hat{a_{ij}}=\displaystyle\frac{a_{ij}u_j}{u_i}$\\
  だから$\sum_{j=1}^n \hat{a_{ij}}=\displaystyle\frac{\sum_{j=1}^n a_{ij}u_j}{u_i}=\frac{{(Au)}_i}{u_i}=\frac{{(\rho (A)u)}_i}{u_i}=\rho (A)$
\end{itemize}

$\hat{A}$についてGershgorinの定理を使うと$r_i = \sum_{j\neq i}\hat{a_{ij}}=\rho (A)-\hat{a_{ii}}$

円板$|\lambda -\hat{a_{ii}}|\leq r_i$とスペクトル円の交わりは$\lambda =\rho (A)$のみである.
\end{proof}
\begin{itembox}[l]{系}
  系1 $A>0$(正行列のみ),既約$\Rightarrow$スペクトル円上の固有値は$\rho (A)$のみ\\
  系2 $\sigma =1\Rightarrow$スペクトル円上の固有値は$\rho (A)$のみ
\end{itembox}

系2の証明は以下のとおりである.

$\sigma=1\Rightarrow$十分に大きなすべての$m$について$A^m >0$

$\lambda$が$A$の固有値で$|\lambda |=\rho (A)$ならば,
${\lambda}^m$は$A^m$
の固有値で$|{\lambda}^m |=\rho (A^m )={\rho (A)}^m$

系1を$A^m$に適用すると${\lambda}^m =\rho (A^m )={\rho (A)}^m$であり$m$は十分大きなすべての数を動くので$\lambda =\rho (A)$
