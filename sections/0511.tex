\subsection{標準固有値問題}
\begin{table}[H]
  \centering
  \begin{tabular}{cc}
    標準固有値問題&$Ax=\lambda x\ \ \ (x\neq 0)$\\
    一般化固有値問題&$Ax=\lambda Gx\ \ \ (x\neq 0)$
  \end{tabular}
\end{table}
ただし一般化固有値問題において$G$は正定値対称行列である.
\subsection{線形空間と双対空間}
線形空間$V$上の線形写像$f$とは,$f:V\to \realnspace{1}$であって以下を満たすものである.
\begin{equation}
  \begin{cases}
    f(x+y) = f(x)+f(y)&\forall x,y\in V\\
    f(\alpha x) = \alpha f(x) & \forall\alpha\in\realnspace{1},\forall x\in V
  \end{cases}
\end{equation}
\begin{dfn}
  $V$の双対空間$V^{\ast}$とは以下のように$V$上の線形写像全体のなす集合である.
  \begin{equation}
    V^{\ast} = \{ f:V\to\realnspace |fは線形写像\}
  \end{equation}
\end{dfn}

双対性を表す内積として以下のように$<,>$をもちいる.
\begin{equation}
  x\in V,\ \ \ f\in V^{\ast}のとき\ \ \ f(x) =\pairing{f}{x}
\end{equation}
各$x\in V$に対して
\begin{align}
  &{\phi}_x :V^{\ast}\to\realnspace{1}\ \ \ (f\in V^{\ast},\pairing{f}{x}\in\realnspace{1})\\
  &{\phi}_x (f)=\pairing{f}{x}\\
  &{\phi}_x \in V^{\ast\ast}\ \ \ ({(V^{\ast})}^{\ast}=V^{\ast\ast}とかく)
\end{align}

このとき$V\to V^{\ast\ast}(x\in V,{\phi}_x \in V^{\ast\ast})$を考えると$V\subseteq V^{\ast\ast}$で一般には$V\neq V^{\ast\ast}$
\begin{align}
  &Vが有限次元空間のとき
  \begin{cases}
    V\simeq \realnspace{n}\\
    V^{\ast}\simeq\realnspace{n}\\
    V^{\ast\ast}\simeq\realnspace{n}
  \end{cases}\\
  &Vがノルム空間のとき
  \begin{cases}
    V^{\ast}は\mathrm{Banach}空間\\
    V^{\ast\ast}も\mathrm{Banach}空間\\
    でもV^{\ast\ast}\simeq Vとは限らない
  \end{cases}\\
  &Vが\mathrm{Hilbert}空間\Rightarrow V\simeq V^{\ast}\simeq V^{\ast\ast}
\end{align}
ここでノルム空間はノルムが定義されたベクトル空間,Banach空間は完備なノルム空間,Hilbert空間はユークリッド空間の拡張である.
\subsection{pairingと内積}
\begin{table}[H]
  \centering
  \begin{tabular}{cc}
    $\pairing{f}{x}$&$\pairing{}{}:V\times V^{\ast}\to\realnspace{1}$\\
    ${(x,y)}_G$&$(,):V\times V\to \realnspace{1}$
  \end{tabular}
\end{table}
ここで正定値対称行列$G:V\to V^{\ast}$を用いると
\begin{align}
  {(x,y)}_G &= \pairing{x}{Gx}\\
  \normsuffix{x}{G}&=\sqrt{\pairing{x}{Gx}}
\end{align}
\section{行列の標準形}
\subsection{対角化}
$A$が実対称行列$\transpose{A}=A$(エルミート行列$A^H = A$,共役転置)のとき固有値は実数で$n$個存在し,固有ベクトルは直交化できる.
\begin{equation}
  \exists 直交行列またはユニタリ行列Qについて\ \ \ Q^{\ast}AQ=
  \begin{pmatrix}
    {\lambda}_1 &\ &0\\
    \ &\ddots&\ \\
    \ &\ &{\lambda}_n
  \end{pmatrix}(Q^{\ast}=\transpose{Q}\mathrm{or}\ Q^{H})
\end{equation}
このように対角化できる行列はどんな行列か?

\begin{dfn}
  \begin{align}
    &Aが正規行列\Leftrightarrow A^{\ast}A = AA^{\ast}\\
    &Aがユニタリ(直交)行列\Leftrightarrow A^{\ast}A=\unitmatrix
  \end{align}
\end{dfn}
\begin{thm}
  \begin{equation}
    Aが正規行列\Leftrightarrow \exists Q(ユニタリ行列)を用いて対角化可能:Q^{\ast}AQ=
    \begin{pmatrix}
      {\lambda}_1 &\ &0\\
      \ &\ddots&\ \\
      \ &\ &{\lambda}_n
    \end{pmatrix}
  \end{equation}
\end{thm}

$\Leftarrow$を示す.対角行列を$D$とすると$DD^{\ast}=D^{\ast}D$であり$A=QDQ^{\ast}$だから$A^{\ast}={(QDQ^{\ast})}^{\ast}={Q^{\ast}}^{\ast}D^{\ast}Q^{\ast}=QD^{\ast}Q^{\ast}$なので計算すれば$AA^{\ast}=A^{\ast}A$が示せる.

$\Rightarrow$は$A$のSchur分解すると対角行列となることを示せば十分.
\begin{thm}
  (正規とは限らない)$A$に対して$\exists Q(ユニタリ):Q^{\ast}AQ$(Schur分解)したものは上三角行列で対角要素は$A$の固有値になる.
\end{thm}
証明は帰納法による.$Ax_1 ={\lambda}_1 x_1 \ \ \ \normsuffix{x_1}{}=1$ととる.このとき$Q=(x_1 |U)$とする.$U^{\ast}x_1 =0(Uの各列はx_1 と直交しているので)$
\begin{equation}
  AQ=\begin{array}{|c|c|}
    \hline
    \lambda x_1&AU\\\hline
\end{array}
\end{equation}
なので
\begin{align}
  Q^{\ast}AQ=
  \begin{pmatrix}
    {x_1}^{\ast}\\
    U^{\ast}
  \end{pmatrix}A(x_1 |U)&=\begin{pmatrix}
    {x_1}^{\ast}\\
    U^{\ast}
  \end{pmatrix}({\lambda}_1 x_1 |AU)\\
  &=
  \begin{pmatrix}
    {x_1}^{\ast}\lambda x_1 & {x_1}^{\ast}AU\\
    U^{\ast}{\lambda}_1 x_1 &U^{\ast}AU
  \end{pmatrix}\\
  &=
  \begin{array}{|c|c|}
    \hline
    {\lambda}_1 &{x_1}^{\ast}AU\\\hline
    0&U^{\ast}AU\\\hline
  \end{array}
\end{align}
最後に$\normsuffix{x}{}=1およびU^{\ast}x_1 =0$を用いた.これより左下の成分は0であり繰り返せば上三角行列になることが示される.

$A$が正規行列($A^{\ast}A=AA^{\ast}$)ならばSchur分解した$Q^{\ast}AQ=R$の$R$も正規である($R^{\ast}R=RR^{\ast}$)(計算すると$RR^{\ast}=R^{\ast}R=Q^{\ast}AA^{\ast}Q$になるので).したがって$RR^{\ast}=R^{\ast}R$を書き下すと
\begin{equation}
  \begin{array}{|ccc|}
    \hline
    r_{11}&\cdots&r_{1n}\\
    &&\\
    0&&r_{nn}\\\hline
  \end{array}\
  \begin{array}{|ccc|}
    \hline
    \conjugate{r_{11}}&&0\\
    \vdots&&\\
    \conjugate{r_{1n}}&&\conjugate{r_{nn}}\\\hline
  \end{array}=
  \begin{array}{|ccc|}
    \hline
    \conjugate{r_{11}}&&0\\
    \vdots&&\\
    \conjugate{r_{1n}}&&\conjugate{r_{nn}}\\\hline
  \end{array}\
  \begin{array}{|ccc|}
    \hline
    r_{11}&\cdots&r_{1n}\\
    &&\\
    0&&r_{nn}\\\hline
  \end{array}
\end{equation}
上の式で(1,1)成分を取り出すと${\normsuffix{r_{11}}{}}^2 +\cdots +{\normsuffix{r_{1n}}{}}^2 = {\normsuffix{r_{11}}{}}^2$になるので${\normsuffix{r_{12}}{}}^2 + \cdots +{\normsuffix{r_{1n}}{}}^2 =0$である.したがって$R$は対角行列である.
\subsection{いろいろな分解}
\begin{align}
  &Aがn\times n行列\Leftrightarrow\exists Q(ユニタリ行列)についてQ^{\ast}AQ=上三角行列({\mathrm{Schur}}分解)\\
  &Aが正規行列\Leftrightarrow\exists Q(ユニタリ行列)についてQ^{\ast}AQ=
  \begin{pmatrix}
    {\lambda}_1 &&\\
    &\ddots&\\
    &&{\lambda}_n
  \end{pmatrix}(固有値分解)\\
  &Aのn個の固有ベクトルが独立\Leftrightarrow\exists X(正則)について\inverse{X}AX=
  \begin{pmatrix}
    {\lambda}_1 &&\\
    &\ddots&\\
    &&{\lambda}_n
  \end{pmatrix}\\
  &Aがエルミート行列(実対称)\Leftrightarrow\exists Q(ユニタリ行列)についてQ^{\ast}AQ=
  \begin{pmatrix}
    {\lambda}_1 &&\\
    &\ddots&\\
    &&{\lambda}_n
  \end{pmatrix}(固有値は実数)
\end{align}
\subsection{Sylvester標準形}
\subsubsection{定義}
正則行列$S$を用いてSylvester標準形は以下のように定義される.
\begin{equation}
  A:エルミート行列に対して\exists S(正則行列)を用いてS^{\ast}AS=
  \begin{array}{|c|c|}
    \hline
    \begin{array}{cccccc}
      1&&&&&0\\
      &1&&&&\\
      &&1&&&\\
      &&&1&&\\
      &&&&-1&\\
      0&&&&&-1
    \end{array}&0\\\hline
    0&0\\\hline
  \end{array}
\end{equation}
1の個数を$s$,-1の個数を$t$とすると$s+t=\mathrm{rank}A$である.
\begin{thm}
  $(s,t,n-s-t)$をSylvesterの符号指数と定義すると,以下のSylvesterの慣性則が成り立つ.\\
  Sylvesterの慣性則:$S^{\ast}AS(Sは正則)$の形の変換で符号指数が変わらない.
\end{thm}
\subsubsection{作り方}
固有値分解して$U^{\ast}AU=
\left( \begin{array}{c|c}
  \begin{array}{cccccc}
    {\lambda}_1&&&&&0\\
    &\ddots&&&&\\
    &&{\lambda}_s&&&\\
    &&&{\lambda}_{s+1}&&\\
    &&&&\ddots&\\
    0&&&&&{\lambda}_{n}
  \end{array}&0\\\hline
  0&0
\end{array}\right)$となったとき(${\lambda}_1 \cdots {\lambda}_s >0\ \ \ {\lambda}_{s+1}\cdots {\lambda}_n <0$)以下のように$S$を定めると$S^{\ast}AS$はSylvester標準形になる.
\begin{equation}
  S=
  \left( \begin{array}{c|c}
    \begin{array}{ccc}
      1/\sqrt{{\lambda}_1}&&\\
      &\ddots&\\
      &&1/\sqrt{{\lambda}_n}
    \end{array}&0\\\hline
    0&
    \begin{array}{ccc}
      1&&\\
      &\ddots&\\
      &&1
    \end{array}
  \end{array}\right)
\end{equation}
