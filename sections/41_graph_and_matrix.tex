\setcounter{section}{3}
\section{グラフと行列}
$A=(a_{ij})(n\times n行列)$について
\begin{align}
  &V=\{ 1,2,\cdots n\}\\
  &E=\{ (i,j)| a_{ij}\neq 0\}
\end{align}

\begin{align}
  Gが強連結(任意の二点間に道がある)
  &\Leftrightarrow Aが既約\\
  &\Leftrightarrow PA\transpose{P}(P:置換行列
  \footnotemark)=
  \begin{array}{|c|c|}
    \hline
    &\\\hline
    0&\\\hline
  \end{array}\ にできない
\end{align}
\footnotetext{各行各列にちょうど一つだけ 1 の要素を持ち,それ以外は全て0となるような二値正方行列}

2部グラフのマッチングとはいくつかの枝の集合で,
行同士列同士がぶつからない,
つまり端点を共有しない枝の集合.
$M:マッチング$としたとき$\mathrm{argmax}_{M;\mathrm{matching}} |M|$を最大マッチングという.

また,任意の枝$e$に対して,
$(eの左端点)\in X$または$(eの右端点)\in Y$となる
$\Leftrightarrow$
$(X,Y)$を被覆(カバー)という.
\subsection{Koning-Egervaryの定理}
\begin{itembox}[l]{Koning-Egervaryの定理}
\begin{equation}
  \max_{M:\mathrm{matcing}} |M|=\min_{(X,Y):\mathrm{cover}} (|X|+|Y|)
\end{equation}
ここで$M$はマッチングで$(X,Y)$はカバーである.
\end{itembox}
この定理からいえることとして
\begin{itemize}
  \item[(1)]任意のマッチング$M$と任意のカバー$(X,Y)$に対して$|M|\leq |X|+|Y|$\\
  $(X,Y)$はカバーであることから,両方ともに属さない枝は存在しない.
  \item[(2)]あるマッチングとあるカバーに対して$|M|=|X|+|Y|$.このとき最大マッチングである.

\end{itemize}


\subsection{Hall-Oreの定理}
\begin{itembox}[l]{Hall-Oreの定理}
$U,V$の二部グラフを考える.
$\Gamma (X)=\{ j\in V| {}^{\exists}i\in X:(i,j)\in E\} (\Gamma :2^U \to 2^V )$
\begin{equation}
  \max_{M;\mathrm{matching}} |M|=\min_{X\subseteq U}(|\Gamma (X)|-|X|)+|U|
\end{equation}
系.$|U|=|V|$のとき以下も成り立つ.
\begin{equation}
  完全マッチングが存在する\Leftrightarrow \forall X \subseteq Uに対して|\Gamma (X)|\geq |X|
\end{equation}
\end{itembox}
\begin{proof}
$U$のなかで$X$以外の残りを$A, \Gamma (X)=B$と表記する.
\begin{align}
  |A|+|X|&=|U|\\
  |B|&=|\Gamma (X)|\\
  |A|+|B|&=|\Gamma (X)|-|X|+|U|
\end{align}
ここで$(A,B)$はカバーでありKoningの定理より$\max |M|=\min (|A|+|B|)$なので示された.\\
系については,
$|M|=|U|$ならば完全マッチングであり
\begin{equation}
  0\geq \max |M|-|U|=\min (|\Gamma (X)|-|X|)
\end{equation}
したがって$U$の部分集合となるすべての$X$に対して
$|\Gamma (X)|-|X|\geq 0$.
\end{proof}

