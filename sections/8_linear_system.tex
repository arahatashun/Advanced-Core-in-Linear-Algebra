\section{線形システム}
\subsection{初期値問題}
\begin{itembox}[l]{線形微分方程式}
以下の線形微分方程式系を考えると解は$x(t)=e^{tA}x^0 (t\geq 0)$となる.
\begin{equation}
  \dot{x}(t)=Ax(t),\ x(0)=x^0 (A\in\realnspace{n\times n})
\end{equation}
\end{itembox}

\begin{dfn}{行列の指数関数}
  $e^A = \displaystyle\sum_{k=0}^{\infty}\frac{1}{k!}A^k$
\end{dfn}
\begin{thm}
  $\lim_{r\to\infty}\displaystyle\sum_{k=0}^r \frac{1}{k!}A^k$は収束する
\end{thm}
なぜなら$S_r = \displaystyle\sum_{k=0}^r \frac{1}{k!}A^k$とすると
\begin{equation}
  \normsuffix{e^A -S_r}{}=\normsuffix{\displaystyle\sum_{k=r+1}^{\infty}\frac{1}{k!}A^k}{}\leq\sum_{r=k+1}^{\infty}\frac{1}{k!}A^k = e^{\normsuffix{A}{}}-\sum_{k=0}^r \frac{1}{k!}{\normsuffix{A}{}}^k \to 0\ (\normsuffix{PQ}{}\leq\normsuffix{P}{}\normsuffix{Q}{})
\end{equation}
\begin{thm}
\begin{equation}
    e^{tA}=\int_0^t Ae^{\tau A}\diff\tau +I \text{ したがって, }\displaystyle\frac{\diff}{\diff t}e^{At}=Ae^{tA}=e^{tA}A
\end{equation}
\end{thm}

証明は以下.
\begin{align}
  \displaystyle\int_0^t Ae^{\tau A}\diff\tau &=\int_0^t A\left( \sum_{k=0}^{\infty}\frac{1}{k!}{(\tau A)}^k\right) \diff\tau\\
  &=\displaystyle\int_0^t \sum_{k=0}^{\infty}{\tau}^k \frac{1}{k!}A^{k+1}\diff\tau\\
  &=\displaystyle\sum_{k=0}^{\infty}\frac{t^{k+1}}{k+1}\frac{A^{k+1}}{k!}\\
  &=\displaystyle\sum_{k=0}^{\infty}\frac{1}{(k+1)!}{(tA)}^{k+1}\\
  &=\displaystyle\sum_{k'=0}^{\infty}\frac{1}{k'!}{(tA)}^{k'}-I\left( \frac{1}{0!}{(tA)}^0 \right)
\end{align}
\subsection{AのJordan標準形}
\begin{itembox}[l]{Jordan標準形を用いた表現}
正則行列$S$を用いて$J=\inverse{S}AS$を考えると
\begin{equation}
    e^{tA} = Se^{tJ}S^{-1}
\end{equation}
\end{itembox}

\begin{align}
  &\inverse{S}e^{tA}S=\displaystyle\sum_{k=0}^{\infty}\frac{1}{k!}\inverse{S}{(tA)}^k S=\sum_{k=0}^{\infty}\frac{1}{k!}t^k {(\inverse{S}AS)}^k = e^{tJ}\\
  &e^{tA}=Se^{tJ}\inverse{S}
\end{align}

$A=\lambda\unitmatrix +J_4 (0)=
\begin{pmatrix}
  \lambda&1&0&0\\
  0&\lambda&1&0\\
  0&0&\lambda&1\\
  0&0&0&\lambda
\end{pmatrix}$を考えてみる.
\begin{align}
  e^{tA}&=e^{t\lambda I}e^{tJ_4 (0)}\\
  &=(e^{t\lambda}\unitmatrix )e^{tJ_4 (0)}
 \end{align}
 $J(0)$はべき零.
 \begin{align}
  e^{tJ_4 (0)}&=\sum_{k=0}^3 \frac{1}{k!}t^k {J_4 (0)}^k\\
  &=\unitmatrix +tJ_4 (0)+\displaystyle\frac{1}{2}t^2 {J_4 (0)}^2 +\frac{1}{6}t^3 {J_4 (0)}^3\\
  &= \begin{bmatrix}
    1 & t & t^2/2 & t^3/6\\
    0 & 1 & t & t^2/2\\
    0 & 0 & 1 & t\\
    0 & 0 & 0 & 1
  \end{bmatrix}
\end{align}
\subsection{安定性}
\begin{itembox}[l]{安定性}
\begin{align}
    \forall x_0 \in \mathbb{R}^n \quad 
    & x(t) = e^{tA}x_o \rightarrow 0 \quad (t\rightarrow \infty)\\
    &\Updownarrow\\
    e^{t\lambda} \text{が支配的になるので,}&A\text{のすべての固有値について}
    \mathrm{Re} \lambda < 0
\end{align}
これを$A$が安定という.
\end{itembox}
\subsection{Lyapunov不等式}
\begin{itembox}[l]{Lyapunov不等式}
  $A(n\times n実行列)$が安定\\
  $\Leftrightarrow \exists Y(n\times n)$について,
  \begin{align}
    &Y=\transpose{Y} \succ 0\ (正定値 対称行列)\\
    &YA+\transpose{A}Y \prec 0\ (負定値)
  \end{align}
負定値の場合の不等式をLyapunov不等式という.
\end{itembox}
\begin{proof}
$\Leftarrow$は容易で$\lambda$を$A$の固有値とすると
\begin{align}
  &Av=\lambda v(v\neq 0)\\
  &v^{\ast}\transpose{A}=\overline{\lambda}v^{\ast}\ (共役転置)
\end{align}

$YA+\transpose{A}Y \prec 0$なので
\begin{align}
  &v^{\ast}(YA+\transpose{A}Y)v<0\\
  &v^{\ast}Y\underbrace{Av}_{\lambda v}+\underbrace{v^{\ast}\transpose{A}}_{\overline{\lambda}v^{\ast}}Yv<0\\
  &(\lambda +\overline{\lambda})v^{\ast}Yv<0
 \end{align}
よって
\begin{equation}
     2{\mathrm{Re}}\lambda = \lambda +\overline{\lambda}<0
      \quad (\because Y \succ 0)
\end{equation}
$\Rightarrow$について${\mathrm{Re}}\lambda <0$なので$e^{tA}\to 0(t\to\infty )$である.

正定値実対称行列$Q \succ 0$を用いて以下のように置けばよい.
\begin{equation}
  Y=\displaystyle\int_0^{\infty}e^{t\transpose{A}}Qe^{tA}\diff t 
  \succ 0
\end{equation}
というのも
\begin{align}
  \frac{\diff}{\diff t}(e^{t\transpose{A}}Qe^{tA})&
  =e^{t\transpose{A}}Qe^{tA}A+
  \transpose{A}e^{t\transpose{A}}Qe^{tA}\\
  \int_0^{\infty}\frac{\diff}{\diff t}(e^{t\transpose{A}}Qe^{tA})dt &=YA+\transpose{A}Y\\
  \text{左辺は}\left[e^{t\transpose{A}}Qe^{tA}\right]_0^{\infty} &=0-Q = -Q\\
  -Q &= YA + A^TY \prec 0
\end{align}
\end{proof}
\begin{itembox}[l]{Lyapunov方程式}
\begin{equation}
    AX + XA^T = - Q
\end{equation}
$A,Q$は実正方行列で,$Q$は正定値対称行列とする
\begin{enumerate}
    \item $A$が安定対称ならば解$X$は一意に定まり,正定値対称である.
    \item 正定値対称な解$X$が存在するならば$A$は安定行列である.
\end{enumerate}
\end{itembox}
\subsection{Sylvester方程式}
\begin{itembox}[l]{Sylvester方程式}
$A, B$は正方行列である.
$A:m\times m$行列,
$B:n\times n$行列,
$C:m\times n$行列,
$ X:m\times n$行列であって
未知である.
\begin{equation}
    AX - XB = C
\end{equation}
\end{itembox}
$X$の要素$x_{ij}$に関して線形であるから,
この方程式は$mn$個の要素$x_{ij}$を未知数とする線形方程式系であ
る.これを
$X=\begin{bmatrix}x_1 |x_2 |\cdots x_n\end{bmatrix} \to \tilde{x}=
\begin{bmatrix}
  x_1\\
  x_2\\
  \vdots\\
  x_n
\end{bmatrix}(mn\times 1行列)$とおく.Sylvester方程式は以下のように書ける.
\begin{align}
  \left(
  \begin{bmatrix}
    A&&&0\\
    &A&&\\
    &&\ddots&\\
    0&&&A
  \end{bmatrix}-
  \begin{bmatrix}
    b_{11}I_m&b_{21} I_m&0\\
    b_{12} I_m &b_{22} I_m &&\\
    &&\ddots&\\
    0&&&b_{nn} I_m
  \end{bmatrix}\right) \tilde{x}=
  \begin{bmatrix}
    c_1\\
    c_2\\
    \vdots\\
    c_n
  \end{bmatrix}=\tilde{c}
\end{align}
\begin{itembox}[l]{Kronecker積}
$\otimes$はKronecker積を表し
\begin{align}
  &P:m\times n行列,\ \ \ Q:k\times l行列\\
  &P\otimes Q=
  \begin{pmatrix}
    P_{11}Q&P_{12}Q&&P_{1n}Q\\
    P_{21}Q&P_{22}Q&&\\
    &&\ddots&\\
    P_{m1}Q&&&P_{mn}Q
  \end{pmatrix}
  \quad (mk \times nl)
\end{align}
また以下が成り立つ
\begin{equation}
    (AB)\otimes(CD) =
    (A \otimes C) (B \otimes D)
\end{equation}
\end{itembox}
つまり$(I_n \otimes A-\transpose{B}\otimes I_m )\tilde{x}=\tilde{c}$である.
\begin{align}
  &AX-XB=Cが任意のCに対して解を持つ\\
  \Leftrightarrow&(I_n \otimes A-\transpose{B}\otimes I_m )が正則\\
  \Leftrightarrow&(I_n \otimes A-\transpose{B}\otimes I_m )が0を固有値に持たない\\
  \Leftrightarrow&(実は)Aと\transpose{B}が同じ固有値を持たない
\end{align}
\begin{itembox}[l]{解の存在}
Sylvester方程式が任意の$C$に対して一意解をもつための必要十
分条件は ,$A$と$B$が共通の固有値をもたないことである.
\end{itembox}
\begin{proof}
$\Rightarrow$
\begin{equation}
  \begin{cases}
    Au=\alpha u\\
    \transpose{B}v=\beta v
  \end{cases}
  \ \ \ とするとv\otimes u=
  \begin{bmatrix}
    v_1 u\\
    v_2 u\\
    \vdots\\
    v_n u
  \end{bmatrix}
\end{equation}
\begin{align}
  &(I_n \otimes A)(v\otimes u)=v\otimes (Au)=\alpha (v\otimes u)\\
  &(\transpose{B}\otimes I_m )(v\otimes u)=(\transpose{B}v)\otimes u=\beta (v\otimes u)
 \end{align}
 したがって
\begin{equation}
    (I_n \otimes A-\transpose{B}\otimes I_m) (v\otimes u)=
    (\alpha - \beta )(v\otimes u)
\end{equation}
 この式は$\alpha - \beta$が固有値になることを示している.\\
($\Leftarrow$)
Jordan標準形およびSchur分解によって
\begin{align}
    \exists \quad S \quad (正則行列)\qquad  &
    S^{-1}AS = m\times m 上三角行列
    \\
    \exists \quad T \quad (正則行列) \qquad &
    T^{-1}BT = n\times n 下三角行列
\end{align}
Sylvesterの等式
\begin{align}
    & AX -XB = C\\
    & \Leftrightarrow (S^{-1}AS)(S^{-1}XT)
    - (S^{-1}XT)(T^{-1}BT) =  S^{-1}CT
\end{align}
したがって,$A$は上三角行列,$B$は下三角行列としてよい.
このとき
$(I_n \otimes A-\transpose{B}\otimes I_m )$は上三角行列となる.三角行列の固有値は対角成分に等しいので,
任意の$i,j$に対して$\alpha_{ii}-
\beta_{jj} \neq 0$ならば,この行列は正則である.
\end{proof}
例)
\begin{align}
    A &= \begin{bmatrix}
      a_{11} & a_{12}\\
      0 & a_{22}
    \end{bmatrix}\\
    B &= \begin{bmatrix}
      b_{11} & 0 & 0\\
      b_{21} & b_{22} & 0\\
      b_{31} & b_{32} & b_{33}
    \end{bmatrix} 
\end{align}
このとき対角成分はそれぞれの固有値となる.
\begin{equation}
    I_3 \otimes A = \begin{bmatrix}
      A & &\\
      & A &&\\
      &  & A &\\
    \end{bmatrix}
\end{equation}
これは上三角行列となる.
\begin{equation}
    B^T \otimes I_2 =
    \begin{bmatrix}
      b_{11} I_2 & b_{21}I_2 & b_{31}I_2\\
      0 &b_{22} I_2 & b_{32}I_2\\
      0 & 0 & b_{33} I_2
    \end{bmatrix}
\end{equation}

したがって,
$I \otimes A - B^T \otimes I$の対角項は
$a_{ii} - b_{jj}$が並ぶ.
よって,$a_{ii} \neq b_{jj} \rightarrow a_{ii}-b_{jj} \neq 0$
\subsection{Lyapunovv方程式の可解性}
Lyapunov方程式
\begin{equation}
    YA + A^TY + Q = 0
\end{equation}
と,Sylvester方程式
\begin{equation}
    \bar{A}X - X \bar{B} = C
\end{equation}
との対応は,
\begin{align}
    \bar{A} & \leftarrow A^T\\
    \bar{B} & \leftarrow -A
\end{align}
\begin{align}
\text{$\transpose{A}と-A$が同じ固有値を持たない}
    \Leftrightarrow \forall Q:\mathrm{Lyapuonov}方程式が可解\
\end{align}
そこで,$A$が安定であるとする.\\
\begin{align}
&\Leftrightarrow \mathrm{Re}\lambda <0となる(A^Tの
固有値についても同じ).\\
&\Rightarrow \transpose{A}と-Aは同じ固有値を持たない.\\
&\Rightarrow \mathrm{Lyapunov}方程式は一意にとける.
\end{align}