\section{行列の標準形}
\subsection{対角化}
$A$が実対称行列$\transpose{A}=A$(Hermite行列$A^H = A$,共役転置)のとき固有値は実数で$n$個存在し,固有ベクトルは直交化できる.
\begin{equation}
  \exists 直交行列またはユニタリ行列Qについて\ \ \ Q^{\ast}AQ=
  \begin{pmatrix}
    {\lambda}_1 &\ &0\\
    \ &\ddots&\ \\
    \ &\ &{\lambda}_n
  \end{pmatrix}(Q^{\ast}=\transpose{Q}\mathrm{or}\ Q^{H})
\end{equation}
このように対角化できる行列はどんな行列か?

\begin{itembox}[l]{正規行列のユニタリ行列による対角化}
\begin{align}
    &Aが正規行列\Leftrightarrow A^{\ast}A = AA^{\ast}\\
    &Aがユニタリ(直交)行列\Leftrightarrow A^{\ast}A=\unitmatrix
  \end{align}
  \begin{equation}
    Aが正規行列\Leftrightarrow \exists Q(ユニタリ行列)を用いて対角化可能:Q^{\ast}AQ=
    \begin{pmatrix}
      {\lambda}_1 &\ &0\\
      \ &\ddots&\ \\
      \ &\ &{\lambda}_n
    \end{pmatrix}
  \end{equation}
\end{itembox}

$\Leftarrow$を示す.対角行列を$D$とすると$DD^{\ast}=D^{\ast}D$であり$A=QDQ^{\ast}$だから$A^{\ast}={(QDQ^{\ast})}^{\ast}={Q^{\ast}}^{\ast}D^{\ast}Q^{\ast}=QD^{\ast}Q^{\ast}$なので計算すれば$AA^{\ast}=A^{\ast}A$が示せる.

$\Rightarrow$は$A$のSchur分解すると対角行列となることを示せば十分.
\begin{itembox}[l]{Schur分解}
  (正規とは限らない)$A$に対して$\exists Q(ユニタリ):Q^{\ast}AQ$(Schur分解)したものは上三角行列で対角要素は$A$の固有値になる.
\end{itembox}
\begin{proof}
帰納法による.$Ax_1 ={\lambda}_1 x_1 \ \ \ \normsuffix{x_1}{}=1$ととる.このとき$Q=(x_1 |U)$とする.$U^{\ast}x_1 =0(Uの各列はx_1 と直交しているので)$
\begin{equation}
  AQ=\begin{array}{|c|c|}
    \hline
    \lambda x_1&AU\\\hline
\end{array}
\end{equation}
なので
\begin{align}
  Q^{\ast}AQ=
  \begin{pmatrix}
    {x_1}^{\ast}\\
    U^{\ast}
  \end{pmatrix}A(x_1 |U)&=\begin{pmatrix}
    {x_1}^{\ast}\\
    U^{\ast}
  \end{pmatrix}({\lambda}_1 x_1 |AU)\\
  &=
  \begin{pmatrix}
    {x_1}^{\ast}\lambda x_1 & {x_1}^{\ast}AU\\
    U^{\ast}{\lambda}_1 x_1 &U^{\ast}AU
  \end{pmatrix}\\
  &=
  \begin{array}{|c|c|}
    \hline
    {\lambda}_1 &{x_1}^{\ast}AU\\\hline
    0&U^{\ast}AU\\\hline
  \end{array}
\end{align}
最後に$\normsuffix{x}{}=1およびU^{\ast}x_1 =0$を用いた.これより左下の成分は0であり繰り返せば上三角行列になることが示される.
\end{proof}
$A$が正規行列($A^{\ast}A=AA^{\ast}$)ならばSchur分解した$Q^{\ast}AQ=R$の$R$も正規である($R^{\ast}R=RR^{\ast}$)(計算すると$RR^{\ast}=R^{\ast}R=Q^{\ast}AA^{\ast}Q$になるので).したがって$RR^{\ast}=R^{\ast}R$を書き下すと
\begin{equation}
  \begin{array}{|ccc|}
    \hline
    r_{11}&\cdots&r_{1n}\\
    &&\\
    0&&r_{nn}\\\hline
  \end{array}\
  \begin{array}{|ccc|}
    \hline
    \conjugate{r_{11}}&&0\\
    \vdots&&\\
    \conjugate{r_{1n}}&&\conjugate{r_{nn}}\\\hline
  \end{array}=
  \begin{array}{|ccc|}
    \hline
    \conjugate{r_{11}}&&0\\
    \vdots&&\\
    \conjugate{r_{1n}}&&\conjugate{r_{nn}}\\\hline
  \end{array}\
  \begin{array}{|ccc|}
    \hline
    r_{11}&\cdots&r_{1n}\\
    &&\\
    0&&r_{nn}\\\hline
  \end{array}
\end{equation}
上の式で(1,1)成分を取り出すと${\normsuffix{r_{11}}{}}^2 +\cdots +{\normsuffix{r_{1n}}{}}^2 = {\normsuffix{r_{11}}{}}^2$になるので${\normsuffix{r_{12}}{}}^2 + \cdots +{\normsuffix{r_{1n}}{}}^2 =0$である.したがって$R$は対角行列である.
\subsection{いろいろな分解}
\begin{align}
  &Aがn\times n行列\Leftrightarrow\exists Q(ユニタリ行列)についてQ^{\ast}AQ=上三角行列({\mathrm{Schur}}分解)\\
  &Aが正規行列\Leftrightarrow\exists Q(ユニタリ行列)についてQ^{\ast}AQ=
  \begin{pmatrix}
    {\lambda}_1 &&\\
    &\ddots&\\
    &&{\lambda}_n
  \end{pmatrix}(固有値分解)\\
  &Aのn個の固有ベクトルが独立\Leftrightarrow\exists X(正則)について\inverse{X}AX=
  \begin{pmatrix}
    {\lambda}_1 &&\\
    &\ddots&\\
    &&{\lambda}_n
  \end{pmatrix}\\
  &AがHermite行列(実対称)\Leftrightarrow\exists Q(ユニタリ行列)についてQ^{\ast}AQ=
  \begin{pmatrix}
    {\lambda}_1 &&\\
    &\ddots&\\
    &&{\lambda}_n
  \end{pmatrix}(固有値は実数)
\end{align}
\subsection{Sylvester標準形}
\subsubsection{定義}
正則行列$S$を用いてSylvester標準形は以下のように定義される.
\begin{equation}
  A:\mathrm{Hermite}行列に対して\exists S(正則行列)を用いてS^{\ast}AS=
  \begin{array}{|c|c|}
    \hline
    \begin{array}{cccccc}
      1&&&&&0\\
      &1&&&&\\
      &&1&&&\\
      &&&1&&\\
      &&&&-1&\\
      0&&&&&-1
    \end{array}&0\\\hline
    0&0\\\hline
  \end{array}
\end{equation}
1の個数を$s$,-1の個数を$t$とすると$s+t=\mathrm{rank}A$である.
\begin{itembox}[l]{ Sylvesterの慣性則}
$(s,t,n-s-t)$をSylvesterの符号指数と定義すると,
$S^{\ast}AS(Sは正則)$の形の変換で符号指数が変わらない.
\end{itembox}
\subsubsection{作り方}
固有値分解して$U^{\ast}AU=
\left( \begin{array}{c|c}
  \begin{array}{cccccc}
    {\lambda}_1&&&&&0\\
    &\ddots&&&&\\
    &&{\lambda}_s&&&\\
    &&&{\lambda}_{s+1}&&\\
    &&&&\ddots&\\
    0&&&&&{\lambda}_{n}
  \end{array}&0\\\hline
  0&0
\end{array}\right)$となったとき(${\lambda}_1 \cdots {\lambda}_s >0\ \ \ {\lambda}_{s+1}\cdots {\lambda}_n <0$)以下のように$S$を定めると$S^{\ast}AS$はSylvester標準形になる.
\begin{equation}
  S=
  \left( \begin{array}{c|c}
    \begin{array}{ccc}
      1/\sqrt{{\lambda}_1}&&\\
      &\ddots&\\
      &&1/\sqrt{{\lambda}_n}
    \end{array}&0\\\hline
    0&
    \begin{array}{ccc}
      1&&\\
      &\ddots&\\
      &&1
    \end{array}
  \end{array}\right)
\end{equation}

\subsection{特異値分解}
正方行列は場合によっては対角化できたが,正方行列でない行列でも対角化のようなことがしたい.そのようなときは特異値分解を行う.

$m\times n$行列$A$について直交行列(転置行列と逆行列が等しい)$U(m\times m),V(n\times n)$をとってくると
\begin{equation}
  \transpose{U}AV=
  \begin{array}{c|c}
    \begin{array}{ccc}
      {\sigma}_1 &&\\
      &\ddots&\\
      &&{\sigma}_r
    \end{array}&0\\\hline
    0&0
  \end{array}=\Sigma
\end{equation}
ここで$r=\mathrm{rank}A$であり$\Sigma$は特異値行列である.
\begin{align}
  &A=U\Sigma\transpose{V}\\
  &\transpose{A}=V\transpose{\Sigma}\transpose{U}\\
  &A\transpose{A}=U(\Sigma\transpose{\Sigma})\transpose{U}\\
  &\Sigma\transpose{\Sigma}(m\times m)=
  \begin{array}{|c|c|}
    \hline
    \begin{array}{ccc}
      {{\sigma}_1}^2 &&\\
      &\ddots&\\
      &&{{\sigma}_r}^2
    \end{array}&0\\\hline
    0&0\\\hline
  \end{array}
\end{align}
\begin{align}
  Uの列ベクトルは&A\transpose{A}の固有ベクトル\\
  {{\sigma}_j}^2は&A\transpose{A}の固有値\\
  Vの列ベクトルは&\transpose{A}Aの固有ベクトル\\
  {{\sigma}_j}^2は&\transpose{A}Aの固有値
\end{align}
\subsection{Jordan標準形}
\subsubsection{はじめに}
$A=\fourmatrix{0}{1}{0}{0}\ \ \ B=\fourmatrix{0}{0}{0}{0}$とする.
\begin{align}
  \to&\trace{A}=\trace{B}\\
  &\det{A}=\det{B}\\
  &{\lambda}_j (A)={\lambda}_j (B)
\end{align}

しかし,正則行列$S$をとってきても$\inverse{S}AS=B$とはこの場合できない$\to$Jordan標準形が$A$と$B$で違う.
\subsubsection{Jordan標準形}
Jordan細胞を以下のように$k\times k$行列で固有値$\lambda$の右横だけ成分が1の行列として定義する.
\begin{equation}
  J_k (\lambda )=\fbox{
  $\begin{array}{ccccc}
    \lambda&1&&&0\\
    &\lambda&1&&\\
    &&&\ddots&\\
    &&&&1\\
    0&&&&\lambda
  \end{array}$}
\end{equation}

$n\times n$行列$A$のJordan標準形は

\begin{equation}
  \inverse{S}AS=\left( \begin{array}{cccc}
    \elementinbox{J_{k_1}({\lambda}_1 )}&&&0\\
    &\elementinbox{J_{k_2}({\lambda}_2 )}&&\\
    &&\ddots&\\
    0&&&\elementinbox{J_{k_l}({\lambda}_l )}
  \end{array}\right)=J
\end{equation}
ここで$k_1 +k_2 +\cdots +k_l =n$であり${\lambda}_1 \cdots {\lambda}_l$は$A$の固有値
\begin{itemize}
  \item[(1)]$l$は線形独立な固有ベクトルの数\\
  \item[(2)]$A$が対角化可能$\Leftrightarrow l=n$\\
  \item[(3)]$\lambda$に対する$J_{k_i}({\lambda} )$の数=$\lambda$の幾何学的重複度\\
  \item[(4)]$\lambda$に対する$k_i$の和=$\lambda$の代数的重複度\\
  \item[(5)]$J_{k_i}({\lambda} )$のサイズは$J-\lambda\unitmatrix$のべき乗からわかる.
\end{itemize}
\subsubsection{Jordan標準形の性質の例}
\begin{equation}
  J=
  \begin{array}{|cccc|}
    \hline
    \begin{array}{|ccc|}
      \hline
      2&1&\\
      &2&1\\
      &&2\\\hline
    \end{array}&&&\\
    &
    \begin{array}{|cc|}
      \hline
      2&1\\
      &2\\\hline
    \end{array}&&\\
    &&\begin{array}{|cc|}
      \hline
      2&1\\
      &2\\\hline
    \end{array}&\\
    &&&2\\\hline
  \end{array}
\end{equation}

\begin{equation}
  J_k (0) =
  \left(
  \begin{array}{ccccc}
    0&1&&&\\
    &0&1&&\\
    &&&\ddots&\\
    &&&&1\\
    &&&&0
  \end{array}\right) は\mathrm{nilpotent}(ある正の整数nに対して{(J_k (0))}^n =0になること)
\end{equation}
\begin{align}
  &J-2\unitmatrix = \left(
  \begin{array}{cccc}
    J_3 (0)&&&\\
    &J_2 (0)&&\\
    &&J_2 (0)&\\
    &&&0
  \end{array}\right)=
  \left(\begin{array}{cccccccc}
    0&1&0&0&0&0&0&0\\
    0&0&1&0&0&0&0&0\\
    0&0&0&0&0&0&0&0\\
    0&0&0&0&1&0&0&0\\
    0&0&0&0&0&0&0&0\\
    0&0&0&0&0&0&1&0\\
    0&0&0&0&0&0&0&0\\
    0&0&0&0&0&0&0&0\\
  \end{array}\right)\\
  &{(J-2\unitmatrix)}^2 =
  \left(\begin{array}{cccccccc}
    0&0&1&0&0&0&0&0\\
    0&0&0&0&0&0&0&0\\
    0&0&0&0&0&0&0&0\\
    0&0&0&0&0&0&0&0\\
    0&0&0&0&0&0&0&0\\
    0&0&0&0&0&0&0&0\\
    0&0&0&0&0&0&0&0\\
    0&0&0&0&0&0&0&0\\
  \end{array}\right)\\
  &{(J-2\unitmatrix)}^3=0
\end{align}

Jordan標準形の性質を適用していくと
\begin{itemize}
  \item ${(J-2\unitmatrix)}^3 =0\to$最大ブロックの次数は3\\
  \item $\rank{{(J-2\unitmatrix)}^2}=1\to$次数3のブロックの数は1個\\
  \item $\rank{J-2\unitmatrix}=4\to 4=\underbrace{2}_{次数3のランク}\times\underbrace{1}_{次数3の個数}+\underbrace{1}_{次数2のランク}\times \underbrace{2}_{次数2の個数がわかる}$\\
  \item $J$の次数は8$\to 8=3\times 1+2\times 2+1\times \underbrace{1}_{次数1の個数がわかる}$
\end{itemize}
\subsubsection{Jordan標準形の利用}
常微分方程式の解法に利用できる.
\begin{equation}
  \begin{cases}
    \dot{x}(t)=Ax\\
    x(0)=x_0
  \end{cases}
\end{equation}

$A(n\times n)$のJordan標準形を$J$とすると正則行列$S$を用いて$A=SJ\inverse{S}$
\begin{equation}
  \begin{cases}
    \dot{y}(t)=Jy(t)\\
    y(0)=y_0
  \end{cases}\ \ \
  \begin{cases}
    x(t)=Sy(t)\\
    y_0 = \inverse{S}x_0
  \end{cases}
\end{equation}

$J$が対角行列なら以下のように簡単に解ける.
\begin{align}
  &\dot{y_l}(t)={\lambda}_i y_i (t)(i=1,2\cdots n)\\
  &y_i (t)=y_i (0)e^{{\lambda}_i t}
\end{align}

$J$が対角でない場合$J=J_n (\lambda )$だとする.

最後の行について$\dot{y_n}(t)=\lambda y_n (t)よりy_n (t)=y_n (0)e^{\lambda t}$となる.
\begin{align}
  &\dot{y_{n-1}}(t)=\lambda y_{n-1}(t)+y_n (0)e^{\lambda t}\\
  &y_{n-1}(t) = y_{n-1}(0)+y_n (0)t e^{\lambda t}
\end{align}
最後は特殊解求めるやつをつかう.

